
\documentclass[11pt,twoside]{article}
\usepackage[paper=a4paper,portrait=true,margin=2.5cm,ignorehead,footnotesep=1cm]{geometry}
\usepackage{graphics}
\usepackage{hyperref}
\usepackage{paralist}

\linespread{1.1}
\setlength{\pltopsep}{2.5pt}
\setlength{\plparsep}{2.5pt}
\setlength{\partopsep}{2.5pt}
\setlength{\parskip}{2.5pt}

\title{GENIE Tutorial}
\author{Andy Ridgwell}
\date{\today}

\begin{document}



%=================================================================================================================================
%=== BEGIN DOCUMENT ==============================================================================================================
%=================================================================================================================================

\maketitle



%=================================================================================================================================
%=== CONTENTS ====================================================================================================================
%=================================================================================================================================

\tableofcontents



%=================================================================================================================================
%=== CHAPTERS ====================================================================================================================
%=================================================================================================================================



%---------------------------------------------------------------------------------------------------------------------------------
%--- Introduction ----------------------------------------------------------------------------------------------------------------
%---------------------------------------------------------------------------------------------------------------------------------

\newpage
\section{Introduction}\label{Introduction}

The document is intended to provide practical help in configuring and running experiments using GENIE-1.
\\The first section describes a variety of illustrative configurations and example experiments using.
\\The second section is a set of tutorial exercises to develop hands-on experience in using GENEI-1.



%---------------------------------------------------------------------------------------------------------------------------------
%--- Example GENIE-1 configurations and experiments ------------------------------------------------------------------------------
%---------------------------------------------------------------------------------------------------------------------------------

\newpage
\section{Example GENIE-1 configurations and experiments}\label{Example GENIE-1 configurations and experiments}

A variety of different model configurations and experimental designs have been created for reference and for use as a helpful starting-point (template) in creating model experiments.
The GENIE-1 configurations and experiments can be obtained from:
\vspace{-11pt}\begin{verbatim}mygenie.seao2.org\end{verbatim}\vspace{-11pt}
and consists of 3 components:

\begin{compactenum}
	
	\item A user config file containing the namelist changes required to set up the details of the experiment. Download this to the directory where the user configuration files are kept\footnote{Specified by the 2nd passed parameter to \texttt{old\_rungenie.sh}, i.e., the user (experiment) configuration file directory}.
	
	\item A set of files defining any selected forcings of biogeochemical tracers\footnote{The default is to use no forcings with a forcing specification pointed to by: \texttt{bg\_par\_fordir\_name='\${}RUNTIME\_ROOT/genie-biogem/data/input'}}. The tracer forcing definitions are provided in the form of subdirectories, each containing:
	
	\begin{compactitem}
		\item 3 files defining which forcings are selected, what sort of forcing is required (restoring vs. flux), and restoring time-constants: (\texttt{configure\_forcings\_xxx.dat}).
		\item A series of forcing definition files (\texttt{biogem\_force\_*.dat}) -- 3 files for each biogeochemical forcing selected.
	\end{compactitem}
	The names of the subdirectories correspond to the value of the \texttt{bg\_par\_fordir\_name} namelist parameter set in the user configuration file. If the \texttt{bg\_par\_fordir\_name} namelist parameter is not set in the user config file then the default setting is used (no tracer forcings are selected).
	
	Download these directories to where-ever you want -- but note that their location must be consistent with the value of the \texttt{bg\_par\_fordir\_name} namelist parameter in the user config file.

	\item The \texttt{old\_rungenie.sh} script. All the experiments provided assume the use of this run script. No warranty is provided if you do not use this script ...

\end{compactenum}


The various experiments are:

\begin{compactenum}
	\item \texttt{worbe2\_preindustrial} -- Ocean-only carbon cycle preindustrial spin-up [8-level ocean]
	\\Spin-up of the ocean carbon cycle and climate, based on an 8-level resolution version of the GOLDSTEIN ocean circulation model.
	
	\item \texttt{worjh2\_preindustrial} -- Ocean-only carbon cycle preindustrial spin-up [16-level ocean]
	\\Spin-up of the ocean carbon cycle and climate, based on a 16-level resolution version of the GOLDSTEIN ocean circulation model.
	
	\item \texttt{worbe2\_historical} -- Historical transient [8-level ocean].
	\\Transient simulation from year 0 to year 2001 with atmospheric composition (CO2, D14C, CFC-11, CFC-12) constrained to follow the historical record. Re-start from the end of the preindustrial spin-up.
	
	\item \texttt{worjh2\_historical} -- Historical transient [16-level ocean].
		\\Transient simulation from year 0 to year 2001 with atmospheric composition (CO2, D14C, CFC-11, CFC-12) constrained to follow the historical record. Re-start from the end of the preindustrial spin-up.
	
	\item \texttt{worbe2\_futureemissions} -- Future CO2 emissions experiment [8-level ocean].
		\\Transient simulation from year 2001 to year 3001 with the emission of 2180.5 PgC to the atmosphere, following A1 MESSAGE up to the year 2100 and a linear decline thereafter. Re-start from the end of the historical transient.
	
	\item \texttt{worjh2\_futureemissions} -- Future CO2 emissions experiment [16-level ocean].
		\\Transient simulation from year 2001 to year 3001 with the emission of 2180.5 PgC to the atmosphere, following A1 MESSAGE up to the year 2100 and a linear decline thereafter. Re-start from the end of the historical transient.
	
	\item \texttt{worbe2\_pulseresponse} -- Atmospheric CO2 pulse-response (emissions decay) experiment [8-level ocean].
	\\Transient simulation from year 0 to year 10001 with atmospheric CO2 instantaneously doubled to 556 ppm by means of the emission of 590.2 PgC to the atmosphere within a single year. Re-start from the end of the preindustrial spin-up.
	
	\item \texttt{worjh2\_pulseresponse} -- Atmospheric CO2 pulse-response (emissions decay) experiment [16-level ocean].
	\\Transient simulation from year 0 to year 10001 with atmospheric CO2 instantaneously doubled to 556 ppm by means of the emission of 590.2 PgC to the atmosphere within a single year. Re-start from the end of the preindustrial spin-up.
	
	\item \texttt{worbe2\_preindustrial\_fullCC\_spin1} -- Full marine carbon cycle preindustrial spin-up: 1st stage [8-level ocean]
	\\1st-state spin-up of a closed marine carbon cycle (sediments+weathering) and climate, based on an 8-level resolution version of the GOLDSTEIN ocean circulation model.
	
	\item \texttt{worjh2\_preindustrial\_fullCC\_spin1} -- Full marine carbon cycle preindustrial spin-up: 1st stage [16-level ocean]
	\\1st-state spin-up of a closed marine carbon cycle (sediments+weathering) and climate, based on a 16-level resolution version of the GOLDSTEIN ocean circulation model.
	
	\item \texttt{worbe2\_preindustrial\_fullCC\_spin2} -- Full marine carbon cycle preindustrial spin-up: 2nd stage [8-level ocean]
	\\2nd-stage spin-up of an open marine carbon cycle (sediments+weathering) and climate, based on an 8-level resolution version of the GOLDSTEIN ocean circulation model.
	
	\item \texttt{worjh2\_preindustrial\_fullCC\_spin2} -- Full marine carbon cycle preindustrial spin-up: 2nd stage [16-level ocean]
	\\2nd-stage spin-up of an open marine carbon cycle (sediments+weathering) and climate, based on a 16-level resolution version of the GOLDSTEIN ocean circulation model.
	
	\item \texttt{worjh2\_pulseresponse\_fullCC\_S} -- Atmospheric CO2 pulse-response (emissions decay) experiment; sediment feedback [16-level ocean].
	\\Transient simulation from year 0 to year 10001 with atmospheric CO2 instantaneously doubled to 556 ppm by means of the emission of 590.2 PgC to the atmosphere within a single year. Re-start from the end of the preindustrial, full marine carbon cycle preindustrial spin-up.
	
	\item \texttt{worjh2\_pulseresponse\_fullCC\_CS} -- Atmospheric CO2 pulse-response (emissions decay) experiment; sediment +  climate feedback [16-level ocean].
	\\Transient simulation from year 0 to year 10001 with atmospheric CO2 instantaneously doubled to 556 ppm by means of the emission of 590.2 PgC to the atmosphere within a single year. Re-start from the end of the preindustrial, full marine carbon cycle preindustrial spin-up.
	
	\item \texttt{worjh2\_pulseresponse\_fullCC\_CSWcWs} -- Atmospheric CO2 pulse-response (emissions decay) experiment; sediment + climate + weathering (carbonate+silicate) feedback [16-level ocean].
	\\Transient simulation from year 0 to year 10001 with atmospheric CO2 instantaneously doubled to 556 ppm by means of the emission of 590.2 PgC to the atmosphere within a single year. Re-start from the end of the preindustrial, full marine carbon cycle preindustrial spin-up.
	
	\item \texttt{worjh2\_preindustrial\_FeCdcycling} -- Ocean-only preindustrial spin-up with Fe co-limitation (and Cd cycle) [16-level ocean]
	\\Spin-up of the ocean carbon cycle and climate, based on a 16-level resolution version of the GOLDSTEIN ocean circulation model. Biological export is co-limited by Fe (as well as PO4). Dissolved Cd concentrations in the ocean are prognostically calculated.
	
\end{compactenum}
 

\noindent \textbf{NOTE}: The user config files provided for the various experiments \textbf{may have to be edited} consistent with your local software environment.

\noindent \textbf{NOTE}: Unless otherwise stated, there is \textbf{no feedback between CO2 and climate} in the experimental designs, i.e., a change in atmospheric CO2 predicted by BIOGEM will not affect climate. This can be changed by setting:
\\\texttt{ea\_36=y}
\\Radiative forcing of the EMBM atmospheric module will then follow the relative (log) deviation of CO2 from 278 ppm.


\noindent \textbf{NOTE}: Unless otherwise stated, for all runs except preindustrial spin-ups, there is no 'CO2-calcification' feedback (see: \textit{Ridgwell et al.} [2007b]. Runs using the preindustrial spin-up as a re-start employ the field of CaCO3:POC rain ratio predicted at the end of the preindustrial spin-up. A fixed in time CaCO3:PO3 production ratio (but spatially variable) is set by the namelist parameter specification:
\vspace{-11pt}\begin{verbatim}bg_ctrl_force_CaCO3toPOCrainratio=.true.\end{verbatim}\vspace{-5.5pt}
which the requires a file containing a 2-D field of CaCO3:PO3 export rain ratios to be pointed to by the namelist parameter:
\vspace{-11pt}\begin{verbatim}bg_par_CaCO3toPOCrainratio_file\end{verbatim}\vspace{-5.5pt}
Simply set \texttt{bg\_ctrl\_force\_CaCO3toPOCrainratio} to \texttt{.false.} to obtain CaCO3 production as a function of carbonate ion concentration [\textit{Ridgwell et al.}, 2007b].


\noindent \textbf{NOTE}: The 2- and 3-D data fields and time-series that will be saved are the default selections (see: default namelist parameter listings). Not all the results you want might therefore be saved. Similarly, rather more data than you care to look at might be saved (thus bloating the netCDF fields). Either way, you can adjust the data that is saved by adding new namelist parameter specifications to the user config file. For example:

\begin{compactitem}
	\item To select saving of (2-D) time-slice fields of ocean->sediment fluxes, add:
	\vspace{-5.5pt}\begin{verbatim}bg_ctrl_data_save_slice_fsedocn=.true.\end{verbatim}\vspace{-5.5pt}
	
	\item To de-select saving of the time-series of ocean surface carbonate chemistry, add:
	\vspace{-5.5pt}\begin{verbatim}bg_ctrl_data_save_sig_carbSS=.false.\end{verbatim}		\vspace{-5.5pt}

\end{compactitem}

Refer to the user-manual for more information.

\noindent \textbf{NOTE}: When an example syntax for the command-line launching of a model experiment is given, e.g.:
\vspace{-11pt}\begin{verbatim}./old_rungenie.sh genie_eb_go_gs_ac_bg ~/genie_userconfigs/
worbe2_historical 2001
~/genie_output/genie_eb_go_gs_ac_bg.worbe2_preindustrial
\end{verbatim}\vspace{-5.5pt}
this is all on ONE LINE (although in practice it may wrap on a normal screen width), and the components must be SPACE SEPERATED.
\\Again: \textbf{ONE LINE; SPACE SEPERATED}

\noindent \textbf{TIP}: Each experiment specifies a file containing a list of time-slice years (at which 2-D and 3-D fields will be saved) pointed to by the parameter \texttt{bg\_par\_infile\_slice\_name}. The file containing the time-series years (at which time-series data is saved) is pointed to by the namelist parameter \texttt{bg\_par\_infile\_sig\_name}. To change the frequency and/or timing of data saving for time-slice or time-series data saving, either edit one of the files provided on SVN (some of which are used in the user configs provided), or create a new file and set the relevant namelist parameter equal its name. For instance, if you created a time-slice file called \texttt{my\_timeslices.dat}, you would set:
\vspace{-22pt}\begin{verbatim}bg_par_infile_slice_name='my_timeslices.dat'\end{verbatim}\vspace{-5.5pt}
Note that the time-slice and time-series files must live in the directory specified by the namelist parameter \texttt{bg\_par\_indir\_name}, which by default is set:
\vspace{-11pt}\begin{verbatim}bg_par_indir_name="$RUNTIME_ROOT/genie-biogem/data/input"\end{verbatim}\vspace{-5.5pt}


\noindent \textbf{TIP}: You can use the example experimental configurations provided as a template for your own experiments. To do this, just copy the user config file and rename it. Alter the namelist values contained in it and/or add additional parameter changes from the defaults to the end of the file. Forcing directories can be similarly copied and edited, or they could be used unaltered for a variety of experiments.


%---------------------------------------------------------------------------------------------------------------------------------

\subsection{Preindustrial spin-up [8-lvl ocean]}\label{EXP1}

This experiment constitutes a 10001 year integration of the GENIE-1 climate model (GOLDSTEIN ocean + sea-ice + EMBM atmosphere) together with ocean (and atmosphere) carbon cycle. Climatology is non-seasonal and identical to that described in \textit{Ridgwell et al.} [2007a]. The ocean carbon cycle is modified slightly so as to produce an acceptable deep-sea sedimentary distribution of CaCO3 (see subsequent experiments descriptions) but is otherwise as described in \textit{Ridgwell and Hargreaves} [2007]. During the spin-up, the ocean is forced into equilibrium with Preindustrial atmospheric concentrations of: CO2, d13C of CO2, d14C of CO2, CFC-11, and CFC-12 via a series of storing forcings of atmospheric composition.

\noindent The \textit{user config} namelist file is named\footnote{The model experiment will be assigned the same name as this when using \texttt{old\_rungenie.sh}.}:
\vspace{-11pt}\begin{verbatim}worbe2_preindustrial\end{verbatim}\vspace{-11pt}
and contains the following parameter specifications\footnote{Mostly (but not always) changes from the default. Thus, it would be possible to conduct an identical experiment with slightly fewer namelist specification. Some of the (mainly biological) namelist values are re-defined (identically) for completeness.}:

\begin{compactitem}

	\item \texttt{--- TRACER SELECTION ---}
	\begin{compactenum}
	\item Select the atmospheric (gaseous) tracers (\texttt{gm\_atm\_select\_xx}):
		\\ia\_pCO2 (xx=3), ia\_pCO2\_13C (xx=4), ia\_pCO2\_14C (xx=5), ia\_pO2 (xx=6), ia\_pCFC11 (xx=18), ia\_pCFC12 (xx=19)
		\\ Select the ocean (dissolved) tracers (\texttt{gm\_ocn\_select\_xx}):
		\\io\_DIC (xx=3), io\_DIC\_13C (xx=4), io\_DIC\_14C (xx=5), io\_PO4 (xx=8), io\_O2 (xx=10), io\_ALK (xx=11), io\_DOM\_C (xx=15), io\_DOM\_C\_13C (xx=16), io\_DOM\_C\_14C (xx=17), io\_DOM\_P (xx=20), io\_CFC11 (xx=45), io\_CFC12 (xx=46), io\_Ca (xx=35), io\_Mg (xx=50)
		\\is\_POC (xx=3), is\_POC\_13C (xx=4), is\_POC\_14C (xx=5), is\_POP (xx=8), is\_CaCO3 (xx=14), is\_CaCO3\_13C (xx=15), is\_CaCO3\_14C (xx=16), is\_det (xx=22), is\_ash (xx=32), is\_POC\_frac2 (xx=33), is\_CaCO3\_frac2 (xx=34), is\_CaCO3\_age (xx=36)
		\\ Select the sedimentary (particulate) tracers (\texttt{gm\_sed\_select\_xx}):
		\item Initialize the atmospheric (gaseous) tracers (\texttt{ac\_atm\_init\_xx}).
		Initialize the ocean (dissolved) tracers (\texttt{bg\_ocn\_init\_xx}).
		\item Set the total number of ocean tracers (\texttt{=16}).
	\end{compactenum}
	
	\item \texttt{--- BIOLOGICAL NEW PRODUCTION ---}
	\\ Parameters as described in \textit{Ridgwell and Hargreaves} [2007]
	
	\item \texttt{--- ORGANIC MATTER EXPORT RATIOS ---}
	\\ Parameters as described in \textit{Ridgwell et al.} [2007a]
	
	\item \texttt{--- INORGANIC MATTER EXPORT RATIOS ---}
	\\ Parameters as described in \textit{Ridgwell and Hargreaves} [2007]
	
	\item \texttt{--- REMINERALIZATION ---}
	\\ Parameters as described in \textit{Ridgwell and Hargreaves} [2007], except:
	\\ The initial fractional abundance of the recalcitrant CaCO3 component is set to\footnote{The original parameter value was 0.468}:
\vspace{-5.5pt}\begin{verbatim}bg_par_bio_remin_CaCO3_frac2=0.4325\end{verbatim}\vspace{-5.5pt}
	This is to ensure that greater CaCO3 preservation in the sediments occurs when using the \textit{Archer} [1991] model explicitly (see \textit{Ridgwell} [2007]) rather than via a look-up table (see: \textit{Ridgwell} [2001]) as used in the original parameter calibration of  \textit{Ridgwell and Hargreaves} [2007]. Changes made to the sediment porosity parameterization (see \textit{Ridgwell} [2007]) have also affected the preservation of CaCO3.
	
	\item \texttt{--- MISC ---}
	\\Turning on tracer auditing; setting a closed (ocean+atmosphere) system carbon cycle (the default); no seasonal forcing of any of the climate components:
	\\ \texttt{ea\_dosc=.false.} == non-seasonal EMBM atmosphere
	\\ \texttt{go\_dosc=.false.} == non-seasonal GOLDSTEIN ocean
	\\ \texttt{gs\_dosc=.false.} == non-seasonal sea-ice
	
	\item \texttt{--- FORCINGS ---}
	\\Setting the location of the biogeochemistry forcing configuration directory (\texttt{bg\_par\_fordir\_name}) to:
	\\ \texttt{"\~{}/genie\_forcings/worbe2\_preindustrial"}
	\\ This \textbf{must} be edited to reflect the actual location of this directory on your computer.
	
\end{compactitem}

The \textit{forcings} directory (\texttt{worbe2\_preindustrial}) contains the following:

\begin{compactitem}

	\item Selection of forcings:
	\\ \texttt{configure\_forcings\_atm.dat} == Selection of restoring forcing of:
	\\ia\_pCO2, ia\_pCO2\_13C, ia\_pCO2\_14C, ia\_pCFC11, ia\_pCFC12
	\\Time-constant for all restorings set to 0.1 years.
	\\ \texttt{configure\_forcings\_ocn.dat} == No ocean tracer forcings.
	\\ \texttt{configure\_forcings\_sed.dat} == No sedimentary tracer forcings.
	
	\item Spatial and temporal definition of forcings. Three files associated with each selected forcing\footnote{See: user-manual.}:
	\begin{compactenum}
		\item \texttt{biogem\_force\_restore\_atm\_xxx\_I.dat}
		\item \texttt{biogem\_force\_restore\_atm\_xxx\_II.dat}
		\item \texttt{biogem\_force\_restore\_atm\_xxx\_sig.dat}
	\end{compactenum}
	
\end{compactitem}

This user config is used in conjunction with the following primary (flavor, resolution, etc) configuration file\footnote{Specified by the 1st parameter passed to the \texttt{old\_rungenie.sh} script.}:
\vspace{-11pt}\begin{verbatim}genie_eb_go_gs_ac_bg\end{verbatim}\vspace{-5.5pt}

A command-line (all one line) launching of the (10001 year) model experiment might look something like:
\vspace{-11pt}\begin{verbatim}./old_rungenie.sh genie_eb_go_gs_ac_bg ~/genie_userconfigs/
worbe2_preindustrial 10001\end{verbatim}\vspace{-11pt}
where the \textit{user config} file \texttt{worbe2\_preindustrial} is found in the directory:
\vspace{-11pt}\begin{verbatim}~/genie_config/\end{verbatim}\vspace{-5.5pt}

This experiment will be sufficient to diagnose preindustrial radiocarbon distributions in the deep ocean.


%---------------------------------------------------------------------------------------------------------------------------------

\subsection{Preindustrial spin-up [16-lvl ocean]}\label{EXP2}

This experiment constitutes a 10001 year integration of the GENIE-1 climate model (GOLDSTEIN ocean + sea-ice + EMBM atmosphere) together with ocean (and atmosphere) carbon cycle. The experimental design is identical to that of the 8-level ocean described above. However, there are a number of important differences in the configuration.
A seasonal, 16 vertical level version of the ocean circulation component is employed in the GENIE-1 model. The climatology of this configuration of GENIE-1 has then been calibrated against present-day observations by a multi-objective tuning process, using exactly the same observational data of annual average surface air temperature and humidity, and 3-D ocean temperature and salinity as described in \textit{Hargreaves et al.} [2004]. Temperature diffusion around Antarctica (90-60�S) was reduced by 75\% in the 2-D atmospheric energy balance component to capture some of the relative (seasonal) isolation of the atmosphere in this region. The net effect of these adjustments to the physics calibration is a maximum strength of the Atlantic meridional overturning circulation of 16.5 Sv, average annual SST of 17.4�C, and average annual sea-ice extent of 7.1\% of the total ocean area.

\noindent The \textit{user config} namelist file is named:
\vspace{-5.5pt}\begin{verbatim}worjh2_preindustrial\end{verbatim}\vspace{-5.5pt}
and is used in conjunction with the following primary (flavor, resolution, etc) configuration file\footnote{Specified by the 1st parameter passed to the \texttt{old\_rungenie.sh} script.}:
\vspace{-22pt}\begin{verbatim}genie_eb_go_gs_ac_bg_itfclsd_16l_JH\end{verbatim}\vspace{-5.5pt}


%---------------------------------------------------------------------------------------------------------------------------------

\subsection{Historical transient [8-lvl ocean]}\label{EXP3}

This experiment constitutes a 2001 year transient integration of the GENIE-1 climate model (GOLDSTEIN ocean + sea-ice + EMBM atmosphere) together with ocean (and atmosphere) carbon cycle with atmospheric trace gas composition constrained to follow observed (historical) changes.

Notable changes to parameter definitions as defined in the \textit{user config}:
\vspace{-11pt}\begin{verbatim}worbe2_historical\end{verbatim}\vspace{-11pt}
from the associated spin-up experiment, include:

\begin{compactitem}

	\item Prescription of time-slice and time-series mid-point data saving definitions specific to a year 0 to 2001 historical transient experiment:
	\vspace{-5.5pt}\begin{verbatim}bg_par_infile_slice_name="save_timeslice_historical.dat"
	bg_par_infile_sig_name="save_sig_historical.dat"\end{verbatim}\vspace{-5.5pt}
	
	\item Prescription of tracer forcings comprising: restoring of CO2, D14C, CFC-11, and CFC-12 to historical observations:	\vspace{-5.5pt}\begin{verbatim}bg_par_fordir_name="~/genie_forcings/worbe2_historical"\end{verbatim}\vspace{-5.5pt}
	
	\item Prescription of a fixed CaCO3:POC rain ratio field, created from the results of the spin-up\footnote{See: HOW-TO document.} [ALL ONE LINE; SPACE SEPERATED]:
	\vspace{-5.5pt}\begin{verbatim}bg_ctrl_force_CaCO3toPOCrainratio=.true.
	bg_par_CaCO3toPOCrainratio_file=
	"CaCO3toPOCrainratio_worbe2_preindustrial.dat"\end{verbatim}\vspace{-5.5pt}
	The purpose of making this namelist change is to fix the CaCO3:POC rain ratio at the preindustrial distribution. Otherwise, the production of CaCO3 in the model will respond to ocean acidification via the CO2-calcification feedback (see: \textit{Ridgwell et al.} [2007b]).
	
\end{compactitem}

A command-line (all one line) launching of the (2001 year) model experiment might look something like:
\vspace{-11pt}\begin{verbatim}./old_rungenie.sh genie_eb_go_gs_ac_bg ~/genie_userconfigs/
worbe2_historical 2001
~/genie_output/genie_eb_go_gs_ac_bg.worbe2_preindustrial
\end{verbatim}\vspace{-5.5pt}
where the \textit{user config} file \texttt{worbe2\_historical} is found in the directory
\\\texttt{\~{}/genie\_userconfigs/} and the location of the required preindustrial spin-up experiment as the restart is
\\\texttt{\~{}/genie\_output/genie\_eb\_go\_gs\_ac\_bg.worbe2\_preindustrial}

This experiment will be sufficient to diagnose (year 1994) anthropogenic CO2, CFC-11, and CFC-12 inventories and distributions.
\\The time-slice specification file (given by \texttt{bg\_par\_infile\_slice\_name}) can be edited (or an alternative file prescribed) to obtain tracer distributions for other years.

This experiment need not be run for 2001 years starting at year zero and it could instead be run from just before the Industrial Revolution. For example, to start from year 1700 you must set the start year namelist parameter\footnote{The namelist default is zero.}:
\vspace{-11pt}\begin{verbatim}bg_par_misc_t_start=1700.0\end{verbatim}\vspace{-5.5pt}

\noindent \textbf{NOTE}: This experiment assumes that the climate and ocean carbon cycle have previously been integrated under preindustrial boundary conditions.

\noindent \textbf{NOTE}: There is no feedback between CO2 and climate set in this experimental design. This can be changed by adding:
\vspace{-11pt}\begin{verbatim}ea_36=y\end{verbatim}\vspace{-11pt}
at the end of the \textit{user config}.


%---------------------------------------------------------------------------------------------------------------------------------

\subsection{Historical transient [16-lvl ocean]}\label{EXP4}

This experiment constitutes a 2001 year transient integration of the GENIE-1 climate model (GOLDSTEIN ocean + sea-ice + EMBM atmosphere) together with ocean (and atmosphere) carbon cycle.
\\The experimental design is identical to that of the 8-level ocean described above.

\noindent The \textit{user config} namelist file is named:
\vspace{-5.5pt}\begin{verbatim}worjh2_historical\end{verbatim}\vspace{-5.5pt}
and is used in conjunction with the following \textit{base config} (flavor, resolution, etc):
\vspace{-5.5pt}\begin{verbatim}genie_eb_go_gs_ac_bg_itfclsd_16l_JH\end{verbatim}\vspace{-5.5pt}


%---------------------------------------------------------------------------------------------------------------------------------

\subsection{Future CO2 emissions experiment [8-lvl ocean]}\label{EXP5}

This experiment constitutes a 1000 year transient integration of the GENIE-1 climate model (GOLDSTEIN ocean + sea-ice + EMBM atmosphere) together with ocean (and atmosphere) carbon cycle with atmospheric trace gas composition driven by a pulse release of CO2 to the atmosphere of 1000 PgC over the 1st year of the experiment.

Notable changes to parameter definitions as defined in the \textit{user config}:
\vspace{-11pt}\begin{verbatim}worbe2_future\end{verbatim}\vspace{-11pt}
from the associated spin-up experiment, include:

\begin{compactitem}

	\item Prescription of the start year to be 2001, rather than the default of year zero:
	\vspace{-5.5pt}\begin{verbatim}bg_par_misc_t_start=2001.0\end{verbatim}\vspace{-5.5pt}
	
	\item Prescription of tracer forcings comprising: a 1000 PgC pulse to the atmosphere in year 2001:	\vspace{-5.5pt}\begin{verbatim}bg_par_fordir_name="~/genie_forcings/worbe2_02180PgC_A2"\end{verbatim}\vspace{-5.5pt}
	
	\item Prescription of time-slice and time-series mid-point data saving definitions with the finest data save intervals being closest to year 2001:
	\vspace{-5.5pt}\begin{verbatim}bg_par_infile_slice_name="save_timeslice_future.dat"
	bg_par_infile_sig_name="save_sig_future.dat"\end{verbatim}\vspace{-5.5pt}
	
	\item Prescription of a fixed CaCO3:POC rain ratio field, created from the results of the spin-up as per the historical transient run (the re-start experiment) [ALL ONE LINE; SPACE SEPERATED]:
	\vspace{-5.5pt}\begin{verbatim}bg_ctrl_force_CaCO3toPOCrainratio=.true.
	bg_par_CaCO3toPOCrainratio_file=
	"CaCO3toPOCrainratio_worbe2_preindustrial.dat"\end{verbatim}\vspace{-5.5pt}

\end{compactitem}

A command-line (all one line) launching of the (1000 year) model experiment might look something like [ALL ONE LINE; SPACE SEPERATED]:
\vspace{-11pt}\begin{verbatim}./old_rungenie.sh genie_eb_go_gs_ac_bg ~/genie_userconfigs
worbe2_futureemissions 1000 
~/genie_output/genie_eb_go_gs_ac_bg.worbe2_historical
\end{verbatim}\vspace{-5.5pt}

\textbf{NOTE}: The run length (given at the command line) can be readily altered for a longer or shorter run. e.g., for just a 100 year experiment to run from year 2000 to 2100 [ALL ONE LINE; SPACE SEPERATED]:
\vspace{-11pt}\begin{verbatim}./old_rungenie.sh genie_eb_go_gs_ac_bg ~/genie_userconfigs
worbe2_futureemissions 100 
~/genie_output/genie_eb_go_gs_ac_bg.worbe2_historical
\end{verbatim}\vspace{-11pt}
However, when altering the experiment duration (or start year) it is likely that it will be necessary to edit or replace (by altering the value of the appropriate namelist) the time-slice and/or time-series files.

\noindent \textbf{NOTE}: This experiment assumes that the climate and ocean carbon cycle have previously been integrated under historical atmospheric observations.

\noindent \textbf{NOTE}: There is no feedback between CO2 and climate set in this experimental design. This can be changed by adding:
\vspace{-11pt}\begin{verbatim}ea_36=y\end{verbatim}\vspace{-11pt}
at the end of the user config file.


%---------------------------------------------------------------------------------------------------------------------------------

\subsection{Future CO2 emissions experiment [16-lvl ocean]}\label{EXP6}

This experiment constitutes a 1000 year transient integration of the GENIE-1 climate model (GOLDSTEIN ocean + sea-ice + EMBM atmosphere) together with ocean (and atmosphere) carbon cycle.
\\The experimental design is identical to that of the 8-level ocean described above.

\noindent The \textit{user config} namelist file is named:
\vspace{-5.5pt}\begin{verbatim}worjh2_future\end{verbatim}\vspace{-5.5pt}
and used in conjunction with the following \textit{base config}:
\vspace{-5.5pt}\begin{verbatim}genie_eb_go_gs_ac_bg_itfclsd_16l_JH\end{verbatim}\vspace{-5.5pt}


%---------------------------------------------------------------------------------------------------------------------------------

\subsection{Atmospheric CO2 impulse-response experiment [8-lvl ocean]}\label{EXP7}

This experiment constitutes a 10001 year transient integration of the GENIE-1 climate model (GOLDSTEIN ocean + sea-ice + EMBM atmosphere) together with ocean (and atmosphere) carbon cycle with release of sufficient CO2 to double atmospheric CO2 to 556 ppm.

Notable changes to parameter definitions as defined in the \textit{user config}:
\vspace{-11pt}\begin{verbatim}worbe2_pulseresponse\end{verbatim}\vspace{-11pt}
from the associated spin-up experiment, include:

\begin{compactitem}
	
	\item Prescription of tracer forcings comprising: addition of 590.2 PgC over 1 year, equivalent to a doubling of atmospheric CO2 from 278 to 556 ppm during the first year (although in practice there will be some invasion of CO2 into the ocean during the year of applied CO2 emissions to the atmosphere):	\vspace{-5.5pt}\begin{verbatim}bg_par_fordir_name="~/genie_forcings/worbe2_00590PgC_1yr"\end{verbatim}\vspace{-5.5pt}

\end{compactitem}

A command-line launching of the (10001 year) model experiment might look something like:
\vspace{-5.5pt}\begin{verbatim}./old_rungenie.sh genie_eb_go_gs_ac_bg ~/genie_userconfigs
worbe2_pulseresponse 10001 
~/genie_output/genie_eb_go_gs_ac_bg.worbe2_preindustrial
\end{verbatim}\vspace{-5.5pt}

This experiment will be sufficient to diagnose the future ocean uptake behavior of fossil fuel CO2 in response to a pulse emission (concentration doubling).
\\The run length (given at the command line) can be altered for a longer or shorter run.

\textbf{NOTE}: This experiment assumes that the climate and ocean carbon cycle have previously been integrated under preindustrial boundary conditions.

\textbf{NOTE}: There is no feedback between CO2 and climate set in this experimental design. This can be changed by adding: 
\vspace{-11pt}\begin{verbatim}ea_36=y\end{verbatim}\vspace{-11pt}
at the end of the user config file.


%---------------------------------------------------------------------------------------------------------------------------------

\subsection{Atmospheric CO2 impulse-response experiment [16-lvl ocean]}\label{EXP8}

This experiment constitutes a 10001 year transient integration of the GENIE-1 climate model (GOLDSTEIN ocean + sea-ice + EMBM atmosphere) together with ocean (and atmosphere) carbon cycle.
\\The experimental design is identical to that of the 8-level ocean described above.

\noindent The \textit{user config} namelist file is named:
\vspace{-11pt}\begin{verbatim}worjh2_pulseresponse\end{verbatim}\vspace{-11pt}
and used in conjunction with the following \textit{base config}:
\vspace{-11pt}\begin{verbatim}genie_eb_go_gs_ac_bg_itfclsd_16l_JH\end{verbatim}\vspace{-5.5pt}


%---------------------------------------------------------------------------------------------------------------------------------

\subsection{Sediments spin-up: CLOSED SYSTEM [8-lvl ocean]}\label{EXP9}

The experimental design is for a 25001 year integration of the GENIE-1 climate model (8-level GOLDSTEIN ocean + sea-ice + EMBM atmosphere) together with ocean (and atmosphere), sediment, and terrestrial weathering carbon cycle. BUT, the global carbon cycle is configured as a 'closed' system -- no burial of CaCO3 in deep-sea sediments and with weathering (solute input to the ocean) subtracted from the deep ocean.

\noindent The \textit{user config} namelist file is named:
\vspace{-11pt}\begin{verbatim}worbe2_preindustrial_fullCC_spin1\end{verbatim}\vspace{-11pt}
and used in conjunction with the following \textit{base config}:
\vspace{-11pt}\begin{verbatim}genie_eb_go_gs_ac_bg_sg_rg\end{verbatim}\vspace{-5.5pt}

\noindent The command-line launching of the experiment looks something like:
\vspace{-11pt}\begin{verbatim}./old_rungenie.sh genie_eb_go_gs_ac_bg_sg_rg ~/genie_userconfigs
worbe2_preindustrial_fullCC_spin1 25001 
\end{verbatim}\vspace{-5.5pt}

See the genie HOWTO for more information.

\textbf{NOTE}: The ROKGEM weathering input parameter setting is an initial guess and is simply taken from \textit{Ridgwell} [2007]:
\vspace{-11pt}\begin{verbatim}rg_par_weather_CaCO3=10.00E+12\end{verbatim}\vspace{-11pt}
and thus is unlikely to match predicted CaCO3 burial.


%---------------------------------------------------------------------------------------------------------------------------------

\subsection{Sediments spin-up: CLOSED SYSTEM [16-lvl ocean]}\label{EXP10}

The experimental design is for a 25001 year integration of the GENIE-1 climate model (16-level GOLDSTEIN ocean + sea-ice + EMBM atmosphere) together with ocean (and atmosphere), sediment, and terrestrial weathering carbon cycle. BUT, the global carbon cycle is configured as a 'closed' system -- no burial of CaCO3 in deep-sea sediments and with weathering (solute input to the ocean) subtracted from the deep ocean.

\noindent The \textit{user config} namelist file is named:
\vspace{-11pt}\begin{verbatim}worjh2_preindustrial_fullCC_spin1\end{verbatim}\vspace{-11pt}
and used in conjunction with the following \textit{base config}:
\vspace{-11pt}\begin{verbatim}genie_eb_go_gs_ac_bg_sg_rg_itfclsd_16l_JH\end{verbatim}\vspace{-5.5pt}

\noindent The command-line launching of the experiment looks something like:
\vspace{-11pt}\begin{verbatim}./old_rungenie.sh genie_eb_go_gs_ac_bg_sg_rg_itfclsd_16l_JH ~/genie_userconfigs
worjh2_preindustrial_fullCC_spin1 25001 
\end{verbatim}\vspace{-5.5pt}

See the genie HOWTO for more information.

\textbf{NOTE}: The ROKGEM weathering input parameter setting is an initial guess:
\vspace{-11pt}\begin{verbatim}rg_par_weather_CaCO3=9.00E+12\end{verbatim}\vspace{-11pt}
and unlikely to match predicted CaCO3 burial.


%---------------------------------------------------------------------------------------------------------------------------------

\subsection{Sediments spin-up: OPEN SYSTEM [8-lvl ocean]}\label{EXP11}

The experimental design is for a 50001 year integration of the GENIE-1 climate model (8-level GOLDSTEIN ocean + sea-ice + EMBM atmosphere) together with ocean (and atmosphere), sediment, and terrestrial weathering carbon cycle. The global carbon cycle is configured as an 'open' system with burial and loss of CaCO3 in deep-sea sediments balanced by weathering (solute input to the ocean). The model run uses the \textit{re-start} created by the equivalent 25 kyr 'closed' experiment (described above) as well as the predicted weathering flux.

\noindent The \textit{user config} namelist file is named:
\vspace{-11pt}\begin{verbatim}worbe2_preindustrial_fullCC_spin2\end{verbatim}\vspace{-11pt}
and used in conjunction with the following \textit{base config}:
\vspace{-11pt}\begin{verbatim}genie_eb_go_gs_ac_bg_sg_rg\end{verbatim}\vspace{-5.5pt}

\noindent The command-line launching of the experiment looks something like:
\vspace{-11pt}\begin{verbatim}./old_rungenie.sh genie_eb_go_gs_ac_bg_sg_rg ~/genie_userconfigs
worbe2_preindustrial_fullCC_spin2 50001 
~/genie_output/genie_eb_go_gs_ac_bg_sg_rg.worbe2_preindustrial_fullCC_spin1
\end{verbatim}\vspace{-5.5pt}

See the genie HOWTO for more information.


%---------------------------------------------------------------------------------------------------------------------------------

\subsection{Sediments spin-up: OPEN SYSTEM [16-lvl ocean]}\label{EXP12}

The experimental design is for a 50001 year integration of the GENIE-1 climate model (16-level GOLDSTEIN ocean + sea-ice + EMBM atmosphere) together with ocean (and atmosphere), sediment, and terrestrial weathering carbon cycle. The global carbon cycle is configured as an 'open' system with burial and loss of CaCO3 in deep-sea sediments balanced by weathering (solute input to the ocean). The model run uses the \textit{re-start} created by the equivalent 25 kyr 'closed' experiment (described above) as well as the predicted weathering flux.

\noindent The \textit{user config} namelist file is named:
\vspace{-11pt}\begin{verbatim}worjh2_preindustrial_fullCC_spin2\end{verbatim}\vspace{-11pt}
and used in conjunction with the following \textit{base config}:
\vspace{-11pt}\begin{verbatim}genie_eb_go_gs_ac_bg_sg_rg_itfclsd_16l_JH\end{verbatim}\vspace{-5.5pt}

\noindent The command-line launching of the experiment looks something like:
\vspace{-11pt}\begin{verbatim}./old_rungenie.sh genie_eb_go_gs_ac_bg_sg_rg_itfclsd_16l_JH ~/genie_userconfigs
worjh2_preindustrial_fullCC_spin2 50001 
~/genie_output/genie_eb_go_gs_ac_bg_sg_rg_itfclsd_16l_JH.worjh2_preindustrial_fullCC_spin1
\end{verbatim}\vspace{-5.5pt}

See the genie HOWTO for more information.


%---------------------------------------------------------------------------------------------------------------------------------

\subsection{CO2 pulse-response; sediment feedback [16-lvl]}\label{EXP13}

This experiment constitutes a 10001 year transient integration of the GENIE-1 climate model (GOLDSTEIN ocean + sea-ice + EMBM atmosphere) together with ocean (and atmosphere), sediment, and weathering carbon cycle components,with release of sufficient CO2 to double atmospheric CO2 to 556 ppm.

Notable namelist parameter definitions as defined in the \textit{user config}:
\vspace{-11pt}\begin{verbatim}worjh2_pulseresponse_fullCC_S\end{verbatim}\vspace{-11pt}
include:

\begin{compactitem}
	
	\item Prescription of tracer forcings comprising: addition of 590.2 PgC over 1 year, equivalent to a doubling of atmospheric CO2 from 278 to 556 ppm during the first year (although in practice there will be some invasion of CO2 into the ocean during the year of applied CO2 emissions to the atmosphere) AND prescribed 2D rain flux of non-carbonate particulates to the sediments:	\vspace{-5.5pt}\begin{verbatim}bg_par_fordir_name="~/genie_forcings/worjh2_00590PgC_1yr_detplusopalSED"\end{verbatim}\vspace{-5.5pt}

\end{compactitem}

A command-line launching of the (10001 year) model experiment might look something like:
\vspace{-5.5pt}\begin{verbatim}./old_rungenie.sh genie_eb_go_gs_ac_bg_sg_rg_itfclsd_16l_JH ~/genie_userconfigs
worjh2_pulseresponse_fullCC_S 10001 
~/genie_output/genie_eb_go_gs_ac_bg_sg_rg_itfclsd_16l_JH.worjh2_preindustrial_fullCC_spin2
\end{verbatim}\vspace{-5.5pt}

This experiment will be sufficient to diagnose the future ocean uptake behavior of fossil fuel CO2 in response to a pulse emission (concentration doubling).
\\The run length (given at the command line) can be altered for a longer or shorter run.

\textbf{NOTE}: This experiment assumes that the climate and full (sediments+weathering) carbon cycle has previously been integrated under preindustrial boundary conditions.


%---------------------------------------------------------------------------------------------------------------------------------

\subsection{ CO2 pulse-response; sediment + climate feedbacks [16-lvl]}\label{EXP14}

This experiment constitutes a 10001 year transient integration of the GENIE-1 climate model (GOLDSTEIN ocean + sea-ice + EMBM atmosphere) together with ocean (and atmosphere), sediment, and weathering carbon cycle components,with release of sufficient CO2 to double atmospheric CO2 to 556 ppm.

Notable namelist parameter definitions as defined in the \textit{user config}:
\vspace{-11pt}\begin{verbatim}worjh2_pulseresponse_fullCC_CS\end{verbatim}\vspace{-11pt}
as compared to \texttt{worjh2\_pulseresponse\_fullCC\_S} (above) include:

\begin{compactitem}
	
	\item Inclusion of feedback between CO2 and climate: 
\vspace{-5.5pt}\begin{verbatim}ea_36=y\end{verbatim}\vspace{-5.5pt}

\end{compactitem}


%---------------------------------------------------------------------------------------------------------------------------------

\subsection{CO2 pulse-response; sediment + climate + weathering feedbacks [16-lvl]}\label{EXP15}

This experiment constitutes a 10001 year transient integration of the GENIE-1 climate model (GOLDSTEIN ocean + sea-ice + EMBM atmosphere) together with ocean (and atmosphere), sediment, and weathering carbon cycle components,with release of sufficient CO2 to double atmospheric CO2 to 556 ppm.

Notable namelist parameter definitions as defined in the \textit{user config}:
\vspace{-11pt}\begin{verbatim}worjh2_pulseresponse_fullCC_CSWcWs\end{verbatim}\vspace{-11pt}
as compared to \texttt{worjh2\_pulseresponse\_fullCC\_S} (above) include:

\begin{compactitem}
	
	\item Inclusion of feedback between CO2 and climate: 
\vspace{-5.5pt}\begin{verbatim}ea_36=y\end{verbatim}\vspace{-5.5pt}
	
	\item Inclusion of feedback between carbonate: 
\vspace{-5.5pt}\begin{verbatim}rg_opt_weather_CaCO3=.TRUE.\end{verbatim}\vspace{-5.5pt}
and silicate: 
\vspace{-5.5pt}\begin{verbatim}rg_opt_weather_CaSiO3=.TRUE.\end{verbatim}\vspace{-5.5pt}
weathering feedbacks with climate.

	\item Partitioning of baseline weathering into (equal) carbonate and silicate components: 
\vspace{-5.5pt}\begin{verbatim}rg_par_weather_CaSiO3=4.34741E+12\end{verbatim}\vspace{-5.5pt}
\vspace{-5.5pt}\begin{verbatim}rg_par_weather_CaCO3=4.34741E+12\end{verbatim}\vspace{-5.5pt}
(total weathering as a CaCO3 equivalent of 0.869482E+13 mol yr-1 as before).

	\item Addition of a fixed (time-independent) mantle CO2 out-gassing rate in conjunction with the silicate weathering component: 
\vspace{-5.5pt}\begin{verbatim}rg_par_outgas_CO2=4.34741E+12\end{verbatim}\vspace{-5.5pt}

	\item Specification of a global land surface reference temperature: 
\vspace{-5.5pt}\begin{verbatim}rg_par_weather_T0=7.980\end{verbatim}\vspace{-5.5pt}
This is the baseline temperature, deviations from which will drive modifications of the weathering flux. The reference value of mean global land surface air temperature can be obtained from the BIOGEM time-series file: \texttt{biogem\_series\_misc\_SLT.res}.

\end{compactitem}

\textbf{NOTE}: In this example configuration with weathering feedback, the total bicarbonate (or Ca2+) flux has been divided equally between carbonate (CaCO3) and silicate (CaSiO3) rock weathering. Hence, the values of both \texttt{rg\_par\_weather\_CaSiO3} and \texttt{rg\_par\_weather\_CaCO3} are set equal and to exactly half the value that was assigned to \texttt{rg\_par\_weather\_CaCO3} during the open system spin-up. The carbon flux (as HCO3-) to the ocean must also be partitioned between carbonate and silicate weathering. There is no carbon in silicate rocks; instead, atmospheric CO2 is the source of the carbon in HCO3- during silicate rock weathering. This carbon ultimately must be balanced form mantle CO2 out-gassing. Hence, the mantle CO2 out-gassing rate (mol yr-1), \texttt{rg\_par\_outgas\_CO2}, must equal the baseline silicate weathering rate (\texttt{rg\_par\_weather\_CaCO3}) at steady-state. See: \textit{Ridgwell and Zeebe} [2005] for a review of the global carbonate cycle, or \textit{Ridgwell and Edwards} [2007].


%---------------------------------------------------------------------------------------------------------------------------------

\subsection{Preindustrial spin-up with Fe co-limitation [16-lvl ocean]}\label{EXP16}

This experiment constitutes a 10001 year integration of the GENIE-1 climate model (GOLDSTEIN ocean + sea-ice + EMBM atmosphere) together with ocean (and atmosphere) carbon cycle. The experimental design is identical to that of the 16-level ocean described above, except for the addition of a 2nd nutrient, iron (Fe). And that dissolved cadmium (Cd) concentrations are calculated in the ocean.

\noindent The \textit{user config} namelist file is named:
\vspace{-11pt}\begin{verbatim}worjh2_preindustrial_FeCdcycling\end{verbatim}\vspace{-5.5pt}
Notable namelist parameter changes include:

\begin{compactitem}
	
	\item The most fundamental change is the prescription of a different 'biological' (export production) scheme (under the \textit{user config} file heading \texttt{BIOLOGICAL NEW PRODUCTION}):
\vspace{-5.5pt}\begin{verbatim}bg_par_bio_prodopt="Payal_Cd"\end{verbatim}\vspace{-11pt}
This selects a scheme which includes co-limitation of export production by dissolved Fe availability. It also enables a Cd cycle in the ocean.

	\item Additional tracers in the ocean are chosen (but others, such as CFCs, omitted):
\vspace{-5.5pt}\begin{verbatim}
gm_ocn_select_9=.true.
gm_ocn_select_22=.true.
gm_ocn_select_23=.true.
gm_ocn_select_24=.true.
gm_ocn_select_34=.true.
\end{verbatim}\vspace{-5.5pt}
which select for \texttt{io\_Fe}, \texttt{io\_DOM\_Fe} (iron incorporated in dissolved organic matter), \texttt{io\_FeL} (iron bound to a ligand), \texttt{io\_L} (free ligand) and \texttt{io\_Cd} (dissolved cadmium).

\item The new ocean tracers are then initialized:
\vspace{-5.5pt}\begin{verbatim}
bg_ocn_init_9=0.650E-09
bg_ocn_init_22=0.0
bg_ocn_init_23=0.0
bg_ocn_init_24=1.000E-09
bg_ocn_init_34=0.650E-09
\end{verbatim}\vspace{-5.5pt}
(all in units of mol kg-1).

	\item To enable scavenging of Fe in the ocean as well as Fe incorporation into particulate organic matter, the following particulate (sediment) tracers are selected:
\vspace{-5.5pt}\begin{verbatim}
gm_sed_select_10=.true.
gm_sed_select_13=.true.
gm_sed_select_21=.true.
gm_sed_select_25=.true.
\end{verbatim}\vspace{-5.5pt}
which correspond to: \texttt{is\_POFe} (iron incorporated in particulate organic matter), \texttt{is\_POM\_Fe} (Fe scavenged onto POM), \texttt{is\_CaCO3\_Fe} (Fe scavenged onto CaCO3), and \texttt{is\_det\_Fe} (Fe scavenged onto detrital material).

	\item A half-saturation constant for Fe limitation:
\vspace{-5.5pt}\begin{verbatim}
bg_par_bio_c0_Fe=0.10E-09
\end{verbatim}\vspace{-5.5pt}
(in mol kg-1). Parameter related to PO4 are also adjusted.
	
	\item Parameters to control Fe solubility and scavenging in the ocean:
\vspace{-5.5pt}\begin{verbatim}
bg_par_det_Fe_sol=0.0030
bg_par_det_Fe_sol_exp=0.500
bg_par_scav_Fe_sf_POC=0.850
bg_par_scav_Fe_sf_CaCO3=0.000
bg_par_scav_Fe_sf_opal=0.000
bg_par_scav_Fe_sf_det=0.000
bg_par_scav_fremin=0.0
bg_ctrl_bio_red_fixedFetoC=.false.
bg_par_K_FeL_pP=11.0
\end{verbatim}\vspace{-5.5pt}
See comments in the \textit{user config} file, and/or refer to the Namelist Table for a brief description.
Note that without an opal particulate tracer, the opal-specific scavenging rate scalar (\texttt{bg\_par\_scav\_Fe\_sf\_opal}) is somewhat redundant. The ligand stability constant is also (\texttt{bg\_par\_K\_FeL\_pP}) at its default value.
	
	\item Parameters to control the Cd cycle in the ocean:
\vspace{-5.5pt}\begin{verbatim}
bg_ctrl_bio_red_CdtoC_Felim=t
bg_par_bio_red_CdtoC_Felim_min=2.000E-6
bg_par_bio_red_CdtoC_Felim_max=5.000E-6
\end{verbatim}\vspace{-5.5pt}
which is somewhat experimental currently ... :)
	
\end{compactitem}

This \textit{user config} file is used in conjunction with the following \textit{base config} file:
\vspace{-11pt}\begin{verbatim}genie_eb_go_gs_ac_bg_itfclsd_16l_JH\end{verbatim}\vspace{-5.5pt}

A command-line launching of the (10001 year) model experiment might look something like:
\vspace{-22pt}\begin{verbatim}./old_rungenie.sh genie_eb_go_gs_ac_bg_sg_rg_itfclsd_16l_JH ~/genie_userconfigs
worjh2_preindustrial_FeCdcycling 10001
\end{verbatim}\vspace{-5.5pt}

A full (science rather than technical/configuration) description of the Fe cycle as implemented here as well as the Cd cycle can be found in \textit{Ridgwell et al.} [in prep].



%---------------------------------------------------------------------------------------------------------------------------------
%--- GENIE-1 TUTORIALS -----------------------------------------------------------------------------------------------------------
%---------------------------------------------------------------------------------------------------------------------------------

\newpage
\section{GENIE-1 Tutorials}



%---------------------------------------------------------------------------------------------------------------------------------
%--- Contact Information ---------------------------------------------------------------------------------------------------------
%---------------------------------------------------------------------------------------------------------------------------------

\newpage
\section{Contact Information}

\begin{compactitem}
	\item Andy Ridgwell: \texttt{bandy@seao2.org}
\end{compactitem}



%=================================================================================================================================
%=== END DOCUMENT ================================================================================================================
%=================================================================================================================================

\end{document}
