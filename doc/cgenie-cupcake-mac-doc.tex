%
% cgenie-cupcake-mac-doc.tex
%
% Developed by Gregory J. L. Tourte <g.j.l.tourte@bristol.ac.uk>
% Copyright (c) 2015 School of Geographical Sciences - The University of Bristol
% Licensed under the terms of GNU General Public License.
% All rights reserved.
%
% Changelog:
% 2015-04-15 - created
%
% arara: lualatex: {shell: true, draft: true}
% arara: lualatex: {shell: true}
%% arara: clean: {extensions: [aux, toc, bbl, run.xml, bcf, blg, out]}
%
\RequirePackage[l2tabu,orthodox]{nag}
\pdfpageattr {/Group << /S /Transparency /I true /CS /DeviceRGB>>}
\RequirePackage[2015/01/01]{latexrelease}

\documentclass{scrartcl}

\usepackage{scrhack}
\usepackage{scrpage2}
%\usepackage{mathtools}
\AfterPackage!{hyperref}{\usepackage[nameinlink]{cleveref}}
\usepackage {fontspec,microtype}
\usepackage {typearea,geometry}
%\usepackage {needspace}
%\usepackage {graphicx}
%\DeclareGraphicsExtensions{.pdf,.png,.jpg}
\usepackage {xcolor}
\usepackage {luatextra}
%\usepackage {booktabs}
%\usepackage {quoting}
%\usepackage {tabu}
\PassOptionsToPackage{hyphens,obeyspaces,spaces}{url}
\usepackage {polyglossia}
\setdefaultlanguage[variant=british]{english}
\usepackage{verbatim}
\usepackage{minted}
%\usepackage {siunitx,numprint}
%\ExplSyntaxOn
%\cs_new_eq:NN \siunitx_table_collect_begin:Nn \__siunitx_table_collect_begin:Nn
%\ExplSyntaxOff
\usepackage {enumitem}
\setlist{nosep}
\usepackage [english]{selnolig}
\usepackage [british,calc]{datetime2}
\DTMlangsetup[en-GB]{ord=raise}
\usepackage [autostyle,english=british]{csquotes}
\usepackage {hyperxmp}
\usepackage {hyperref}

% Font definitions
\defaultfontfeatures{Ligatures = {TeX}, Scale = {MatchLowercase}}
%\setmainfont [Numbers = {Proportional,OldStyle}]{Linux Libertine O}
\setmainfont [Numbers = {Proportional,OldStyle}]{Minion Pro}
\setsansfont [Numbers = {Proportional,OldStyle}]{Linux Biolinum O}
\setmonofont[ItalicFont = *, ItalicFeatures = FakeSlant, RawFeature = -tlig;-trep, StylisticSet={1,3}]{Inconsolatazi4-Regular.otf}
%\newfontfamily\liningroman[Numbers={Lining}]{Linux Libertine O}

\usepackage[math-style=TeX]{unicode-math}
\setmathfont{Tex Gyre Pagella Math}


\KOMAoptions{
    fontsize	= 11pt,
    numbers		= noendperiod,
    parskip		= half-,
%	twocolumn,
%	twoside,
	%draft,
	captions	= tableheading,
%	toc		= listofnumbered,
	listof		= leveldown,
%	chapterprefix	= true,
	DIV		= 9,
%	BCOR		= 7mm,
}
\recalctypearea
\pagestyle{scrheadings}

% Add accessibility functions with pdf (>v1.5)
\usepackage {accsupp}
% declare function to hide text from selection in pdf, used for the draft
% watermark and linenos in listing
\DeclareRobustCommand\squelch[1]{%
  \BeginAccSupp{method=plain,ActualText={}}#1\EndAccSupp{}%
}

%\BeforeBeginEnvironment{minted}{\begin{singlespace}}
%\AfterEndEnvironment{minted}{\end{singlespace}}
\renewcommand{\theFancyVerbLine}{
  \fontspec[Numbers = {Lining,Monospaced}]{Minion Pro}\color[rgb]{0.5,0.5,0.5}\scriptsize\squelch{\arabic{FancyVerbLine}}}

\usemintedstyle{autumn}

\setminted{%frame = lines,
%	framesep=2mm,
%	linenos=true,
	numbersep = 5pt,
	%gobble = 2,
%	fontsize = \footnotesize,
%	xleftmargin=20pt,
%	xrightmargin=20pt,
	tabsize=3,
	breaklines = true,
	breaksymbolleft = \squelch{\tiny\ensuremath{\hookrightarrow}},
%	breaksymbolright = \squelch{\tiny\ensuremath{\hookleftarrow}},
%	breaksymbolindentright = 0pt,
%	breaksymbolsepright = 0pt
}

\title{cGENIE (Cupcake v0.3)\\
Mac OS X Instructions}
\author{Gregory J. L. Tourte}

\pdfminorversion=7

%\DTMsetstyle{pdf}
\makeatletter
\hypersetup {
	hidelinks,
	breaklinks,
	linktocpage             = true,
	unicode                 = true,
	%hyperfootnotes          = true,
	bookmarksnumbered       = true,
	bookmarksopen           = true,
	pdfdisplaydoctitle      = true,
	%pdfpagelabels           = false,
	plainpages              = false,
	pdfauthor               = {\@author},
	pdftitle                = {\@title},
	pdfcontactemail		= {g.j.l.tourte@bristol.ac.uk},
	pdfcontactcity		= {Bristol},
	pdfcontactpostcode	= {BS8 1SS},
	pdfcontactcountry	= {United Kingdom},
	pdfcontacturl		= {http://www.bridge.bris.ac.uk/},
	pdfsubject              = {},
	pdfkeywords             = {},
	pdfcopyright		= {Copyright © 2015, Gregory J. L. Tourte},
	pdflicenseurl		= {http://creativecommons.org/licenses/by-nc-sa/4.0/},
	pdflang                 = en-GB,
	pdfencoding             = auto,
	pdfduplex               = DuplexFlipLongEdge,
	pdfprintscaling         = None,
	pdfinfo					= {
		CreationDate={D:20150323103000Z},
%		ModDate={\DTMnow},
	},
	%draft,
}
\makeatother

\DeclareRobustCommand\Cpp{\hbox{C\hspace{-.05em}\raisebox{.2ex}{\textbf{+\kern-.2ex+}}}}

\usepackage [os=mac]{menukeys}
\begin{document}

\maketitle

These instructions have been tested on Mac OS X Yosemite (10.10) but should
work on Mavericks (10.9) as well. However we recommend that you use the latest
version of the OS available as we may not be able to reproduce errors and bugs
you may report.

The following instructions will require you to use the Mac OS X Terminal
application. This can be opened by using the \keys{\cmd+\SPACE} key combination
and searching for `\textit{Terminal}', alternatively it can be found in the
\directory{Macintosh HD/Applications/Utilities} directory. We suggest you pin
the application to your Dock for easy access if you haven't done so already.

\section{System Requirements}

To install and run cGENIE on Mac OS X you will need the following packages
installed:

\begin{itemize}
	\item Apple Xcode
	\item Xcode Command line Tools
	\item Python and extensions
	\item gfortran compiler 
	\item the MacPorts environment
%	\item the HomeBrew packaging system 
%	\item scons
%	\item wget
	\item NetCDF libraries
\end{itemize}

\subsection{Apple Xcode}

Apple Xcode can be downloaded for free from the Apple App Store. More details
can be found at \url{https://developer.apple.com/xcode/downloads/}. XCode
contains the GNU C Compiler (\texttt{gcc}) and most of the other libraries and
tools to allow the compilation of the model.

In addition to Xcode, you will need to install the Xcode Command Line Tools.
These used to be installed by default on older versions of Xcode but are now
distributed separately. Once Xcode is installed, you will need to run the
following command in the terminal:
%You can find instructions on installing this at
%\url{http://railsapps.github.io/xcode-command-line-tools.html}\,\footnote{Future
%versions of this document will reproduce these instructions in case the link
%disapear}.

\mint{console}|$ xcode-select --install| %stopzone 

You can check if Xcode is properly installed using the command:

\mint{console}|$ xcode-select -p| %stopzone

which should contain the following line in the response:

\mint{console}{/Applications/Xcode.app/Contents/Developer} 

While we are checking the environment, you can check that \texttt{gcc} is installed properly:

\begin{minted}{console}
$ gcc --version
Configured with: --prefix=/Applications/Xcode.app/Contents/Developer/usr --with-gxx-include-dir=/usr/include/c++/4.2.1
Apple LLVM version 6.1.0 (clang-602.0.53) (based on LLVM 3.6.0svn)
Target: x86_64-apple-darwin14.3.0
Thread model: posix
\end{minted}

\subsection{Homebrew \& MacPorts}

Homebrew is a package manager for OS X which allows users to download and
install packages found in other UNIX style environment such as Linux and keep
them up to date in a managed way in the sense that one can update packages and
manage dependencies. MacPort has a similar aim but with a slightly different
philosophy. We will not compare these two here as it goes beyond the scope of
this document but we have used both to install software required to run cGENIE
on the Mac.

Homebrew and information about it can be found at \url{http://brew.sh/}. To
install the environment, simply type the following line in the Terminal:

\mint{console}|$ ruby -e "$(curl -fsSL https://raw.githubusercontent.com/Homebrew/install/master/install)"| %stopzone

MacPort and information about it can be found at
\url{https://www.macports.org/} and instructions on how to install it are found
at \url{https://www.macports.org/install.php}. Following the installation, you
should make sure your installation is fully up to date by running:

\mint{console}|$ sudo port -v selfupdate| %stopzone

\subsection{Python}

The install scripts, running script as well as the gui for cGENIE are written
in \texttt{python} and require version 2.x but preferably v2.7.x. Mac OS X
comes with \texttt{python} installed by default since version 10.8 (Mountain
Lion) and appears to satisfies the minimum version requirements (Yosemite comes
with version 2.7.6). However, the version on Yosemite doesn't seems to be quite
adequate for some of the scripts within cGENIE, especially the gui. You will
therefore need to install another version of \texttt{python}. The version
provided by homebrew is known to work (whereas the one from macport is known
not to) so you will need to install the homebrew version even if you already
have the macport package (they are no installed in the same location so should
not clash. If you are happy with the homebrew version, remove the macport one
to simplify things).

\mint{console}|$ brew install python|%stopzone

Once installed, you will need to specify the location of your python environment via the
\verb|$CGENIE_PYTHON| environment variable for cGENIE to use it. Using the homebrew
version of python you need to add the following line to your
\verb|.bash_profile| file:

\mint{bash}|export CGENIE_PYTHON="/usr/local/bin/python2.7"|

You should also run that same line from the command line if you do not wish to
have to open a new terminal.

\begin{minted}{console}
$ echo $CGENIE_PYTHON
/usr/local/bin/python2.7
$ (exec $CGENIE_PYTHON --version)
Python 2.7.10
\end{minted}

The newly developed gui requires additional python modules such as
\texttt{matplotlib}. These modules can be installed as normal python modules,
i.e., you can either use the source tarballs or use a packaging system such as
\texttt{pip}. Using pip, you should just need to do:

\begin{minted}{console}
$ pip install matplotlib
$ pip show matplotlib
---
Metadata-Version: 2.0
Name: matplotlib
Version: 1.4.3
Summary: Python plotting package
Home-page: http://matplotlib.org
Author: John D. Hunter, Michael Droettboom
Author-email: mdroe@stsci.edu
License: BSD
Location: /usr/local/lib/python2.7/site-packages
Requires: numpy, pytz, pyparsing, python-dateutil, nose, six, mock
\end{minted}

Since \texttt{python}, and therefore pip, are now install from homebrew, you
should be able to run all the pip commands as a normal user and the modules
will be installed in \path{/usr/local/lib/python2.7/site-packages}. However,
if you have used the distribution's version of python in the past, some old
version of modules may already be installed in
\path{/Library/Python/2.7/site-packages/} instead, at which point some
user-run \texttt{pip} commands may fail. There is no pretty fix for this, you
will need to run the \texttt{pip} command as root (via \texttt{sudo}). You can
either uninstall the offending package as root and then install or upgrade the
new one as your user to keep the \path{/usr/local} environment as clean as
possible, or simply run \texttt{pip} as root instead bearing in mind that you
may have permission issues later when using homebrew that will need to be fixed
as root.

%\subsection{SCons}
%
%\texttt{SCons} is a software construction tools built in \texttt{python} which role and
%functionality is similar to the \texttt{autotool}\slash \texttt{automake} and
%\texttt{make} tool chain. cGENIE Cupcake does not use \texttt{make} any more and is built
%on \texttt{SCons} instead.
%
%As with previous tools, we will use the package provided by Homebrew:
%
%\mint{console}|$ brew install scons|
%
%You can then test your installations with:
%
%\begin{minted}{console}
%$ scons --version
%SCons by Steven Knight et al.:
%	script: v2.3.4, 2014/09/27 12:51:43, by garyo on lubuntu
%	engine: v2.3.4, 2014/09/27 12:51:43, by garyo on lubuntu
%	engine path: ['/usr/local/Cellar/scons/2.3.4/libexec/scons-local/SCons']
%Copyright (c) 2001 - 2014 The SCons Foundation
%\end{minted}

\subsection{Fortran Compiler (\texttt{gfortran})}

The source of the \texttt{gfortran} compiler has changed since the instructions
for Cupcake v0.2. The reasons for this change is that packages have been
updated on the third party repositories, both macport and homebrew, and none of
them work properly. The homebrew package we were using previously has been
removed in favour of a full \texttt{GCC} install (overriding the system one
although not replacing it). The version of \texttt{GCC} from homebrew is now
5.1.0 and is incompatible with the version of the \texttt{netcdf} package we
will be getting from macport (the homebrew package for that still doesn't
work). The macport package also installs the entire \texttt{GCC} suite but with
different names for the binaries making it harder to use with standardise code.

Since our requirement is the \texttt{netcdf} package from macport, and this
package is built with \texttt{gfortran} 4.9, we need to install this version of
\texttt{gfortran}. The \texttt{GCC} maintainers do provide such a package for
Mac, for both Mavericks (10.9) and Yosemite (10.10). Information can be found
at \url{https://gcc.gnu.org/wiki/GFortranBinaries#MacOS}. For Yosemites, the
packages is currently available from
\url{http://coudert.name/software/gfortran-4.9.2-Yosemite.dmg}. 

Once the packages is installed the \texttt{gfortran} binary should be available
in your \verb|$PATH|. You can check that \texttt{gfortran} is properly
installed by issuing the command:

\begin{minted}{console}
$ gfortran --version
GNU Fortran (GCC) 4.9.2
Copyright (C) 2014 Free Software Foundation, Inc.

GNU Fortran comes with NO WARRANTY, to the extent permitted by law.
You may redistribute copies of GNU Fortran
under the terms of the GNU General Public License.
For more information about these matters, see the file named COPYING
\end{minted} 

\subsection{NetCDF Libraries}

For NetCDF we will use the version available from MacPorts (we have tried using
Homebrew for this in order to minimise the requirements and not mix the
packaging environment but unfortunately, the NetCDF brew package could not be
installed on our test machine. If this changes, we will reevaluate our
instructions.

Although previous versions of cGENIE required the C, \Cpp{} and fortran versions
of the library, this is no longer the case and the \Cpp{} layer is not a
requirement anymore. The following commands  will install the necessary NetCDF
libraries:

\begin{minted}{console}
$ sudo port install netcdf
$ sudo port install netcdf-fortran
\end{minted}

\section{Installing and running cGENIE (cupcake)}

With all the steps described in this document, you should be able to follow the
instructions described in the
\href{run:./cupcake-config-build.pdf}{\texttt{cupcake-config-build.pdf}} file
in this folder to install and run the model.

If you have any issues with running cGENIE (cupcake) on Mac OS X, please report
them at \url{https://github.com/genie-model/cgenie/issues}.

\section{Quirks}

Because so much of the system depends on third party repositories, it may not
always behave as expected. We are aware that MacPorts for example provide
packages for most of the tools we get from Hombrew, however, we have had
reports of failures on systems with MacPorts packages where installing the
Homebrew package was a solution. This is especially valid for python.

We realised that making users change there tool chains is very much a
annoyance. We are trying to test multiple install setup and try to understand
why one tool chain fails where the other doesn't, but we currently lack the
resources to do so quickly.

The information in this document is valid at the time of writing with the
version of packages available at that time. Since we do not control the
packages distributed on either Homebrew or MacPorts, we cannot tell whether
updated versions of these packages will still work. We will endeavour to follow
the releases of the packages with use and test any upgrade but it is not
guaranteed happen in a timely manner. The main victim of that is python, the
version of which is currently hardcoded in the code and is required to be
2.7.9.

\end{document}

% vim: ts=4
% vim600: fdl=0 fdm=marker fdc=3 spl=en_gb spell
