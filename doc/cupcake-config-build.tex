\documentclass[a4paper,10pt,article]{memoir}

\usepackage[a4paper,margin=1in]{geometry}

\usepackage[utf8]{inputenc}
\usepackage{fourier}
\usepackage{amsmath,amssymb}

\usepackage{color}
\definecolor{orange}{rgb}{0.75,0.5,0}
\definecolor{magenta}{rgb}{1,0,1}
\definecolor{cyan}{rgb}{0,1,1}
\definecolor{grey}{rgb}{0.25,0.25,0.25}
\newcommand{\outline}[1]{{\color{grey}{\scriptsize #1}}}
\newcommand{\revnote}[1]{{\color{red}\textit{\textbf{#1}}}}
\newcommand{\note}[1]{{\color{blue}\textit{\textbf{#1}}}}
\newcommand{\citenote}[1]{{\color{orange}{[\textit{\textbf{#1}}]}}}

\usepackage{xspace}

\usepackage{listings}
\usepackage{courier}
\lstset{basicstyle=\tiny\ttfamily,breaklines=true,language=Python}

\usepackage{fancyvrb}
\usepackage{url}

\usepackage{float}
\newfloat{listing}{tbp}{lop}
\floatname{listing}{Listing}

\title{GENIE \texttt{cupcake} Configuration and Build System}
\author{Ian~Ross}
\date{4 March 2015}

\begin{document}

\catcode`~=11    % Make tilde a normal character (danger of weirdness...)
%\catcode`~=13    % Make tilde an active character again

\maketitle

This document describes the new configuration and build system
developed for the \texttt{cupcake} version of the GENIE model.
Documentation is divided into three sections, one for users of the
model, one for those who wish to modify the model and one for those
concerned with maintaining and extending the configuration and build
system infrastructure.

\emph{Throughout this document, shell commands are shown in
  \texttt{typewriter} font.  Commands that extend over several lines
  are marked with lines ending \texttt{...} with the following line
  beginning \texttt{...}.}

%======================================================================
\chapter{For GENIE users}

%----------------------------------------------------------------------
\section{Quick start}

Open a shell in your home directory and do this (this is based on the
current repository setup, which will probably change a little soon):
\begin{verbatim}
  git clone git@github.com:genie-model/cgenie.git
  cd ~/cgenie
  git checkout cgenie.cupcake
  ./setup-cgenie
\end{verbatim}
Once the setup script has installed Python (if it needs to), accept
the defaults for all the questions it asks (just hit return at each
prompt to accept the default).

Once everything is done, you can run some basic tests with:
\begin{verbatim}
  ./tests run basic
\end{verbatim}
and you can configure and run jobs as:
\begin{verbatim}
  ./new-job -b cgenie.eb_go_gs_ac_bg.p0650e.NONE ...
            ... -u LABS/LAB_0.snowball snowball 10
  cd ~/cgenie-jobs/snowball
  ./go run
\end{verbatim}
The job results appear in the \texttt{output} directory in the job
directory.

%----------------------------------------------------------------------
\section{Installation and setup}

To install GENIE, first choose a location for the installation.  On
Linux machines, it's probably best just to install GENIE in your home
directory.  Just clone the main \texttt{cgenie} repository from GitHub
using the command
\begin{verbatim}
  git clone https://github.com/genie-model/cgenie.git
\end{verbatim}
This will produce a new directory called \texttt{cgenie} containing
the model source code and build scripts.

Before using the model, it's necessary to do a little bit of setup.
Go into the new \texttt{cgenie} directory and run the
\texttt{setup-cgenie} script:
\begin{verbatim}
  cd ~/cgenie
  ./setup-cgenie
\end{verbatim}
Depending on the machine you're doing this on, the script may need to
download and install a local copy of the Python programming language,
which can take a little while.  Once this is done, the script will ask
where you want to put a number of things -- it's usually find to just
take the defaults (just hit enter at each of the prompts).  In all of
what follows below, we'll assume that you chose the defaults.  The
things the script asks for are:
\begin{itemize}
  \item{The GENIE root installation directory: unless you know what
    you're doing, accept the \texttt{~/cgenie} default for this.}
  \item{The GENIE data directory (default \texttt{~/cgenie-data})
    where base and user model configurations are stored, along with
    forcing files.}
  \item{The GENIE test directory (default \texttt{~/cgenie-test})
    where GENIE jobs with known good outputs can be stored for use as
    tests -- it's possible to run sets of tests and compare their
    results with the known good values with a single command, which is
    useful for making sure that the model is working.}
  \item{The GENIE jobs directory (default \texttt{~/cgenie-jobs})
    where new GENIE jobs are set up -- the \texttt{new-job} script
    (see next section) sets jobs up here by default.}
  \item{The default model version to use for running jobs.  By
    default, the most recent released version is selected, but you can
    type another version if necessary.  You can also set jobs up to
    use any model version later on.}
\end{itemize}
After providing this information, the setup script will ask whether
you want to download the data and test repositories.  It's usually
best to say yes.

Once the data and test repositories have been downloaded, GENIE is
ready to use.  The setup information is written to a
\texttt{.cgenierc} file in your home directory.  If you ever want to
set the model up afresh, just remove this file and run the
\texttt{setup-cgenie} script again.

To check that the installation has been successful and that the model
works on your machine, you can run some basic test jobs -- in the
\texttt{~/cgenie} directory, just type
\begin{verbatim}
  ./tests run basic
\end{verbatim}
This runs an ocean-atmosphere simulation and an ocean biogeochemistry
simulation using the default model version selected at setup time.

\emph{If you are the first person to use GENIE on your machine, you
  will need to set up a platform file to tell GENIE where to find a
  Fortran compiler, what compilation flags to use, and where to find
  NetCDF libraries -- see Section~\ref{sec:platforms}.}

%----------------------------------------------------------------------
\section{Creating new jobs}

New GENIE jobs are configured using the \texttt{new-job} script in
\texttt{~/cgenie}.  This takes a number of arguments that describe the
job to be set up and produces a job directory under
\texttt{~/cgenie-jobs} containing everything needed to build and run
the model with the selected configuration.  The \texttt{new-job}
script should be run as:
\begin{verbatim}
  new-job [options] job-name run-length
\end{verbatim}
where \texttt{job-name} is the name to be used for the job directory
to be created under \texttt{~/cgenie-jobs} and \texttt{run-length} is
the length of the model run in years.  The possible options for
\texttt{new-job} are as follows (in each case given in both short and
long forms where these exist).  First, there are three options that
control the basic configuration of the model.  In most cases, a base
and a user configuration should be supplied (options \texttt{-b} and
\texttt{-u}).  In some special circumstances, a custom ``full''
configuration may also be used (the \texttt{-c} option).

\begin{verbatim}
  -b BASE_CONFIG   --base-config=BASE_CONFIG
\end{verbatim}
The base model configuration to use -- these are stored in the
\texttt{~/cgenie-data/base-configs} directory.

\begin{verbatim}
  -u USER_CONFIG   --user-config=USER_CONFIG
\end{verbatim}
The user model configuration to apply on top of the base configuration
-- model user configurations are stored in the
\texttt{~/cgenie-data/user-configs} directory.

\begin{verbatim}
  -c CONFIG        --config=CONFIG
\end{verbatim}
Full configuration name (this is mostly used for conversions of
pre-\texttt{cupcake} tests) -- full configurations are stored in the
\texttt{~/cgenie-data/full-configs} directory.

In addition to the configuration file options, the following
additional options may be supplied to \texttt{new-job}:

\begin{verbatim}
  -O     --overwrite
\end{verbatim}
Normally, \texttt{new-job} will not overwrite any existing job of the
requested name.  Supplying the \texttt{-O} flag causes
\texttt{new-job} to delete and replace any existing job with the
requested name.

\begin{verbatim}
  -r RESTART    --restart=RESTART
\end{verbatim}
One GENIE job can be \emph{restarted} from the end of another.  This
option allows for a restart job to be specified.  This must be a job
that has already been run (so that there is output data to use for
restarting the model).

\begin{verbatim}
  --old-restart
\end{verbatim}
It may sometimes be useful to restart from an old pre-\texttt{cupcake}
job.  This flag indicates that the job name supplied to the
\texttt{-r} flag is the name of an old GENIE job whose output can be
found in the \texttt{~/cgenie\_output} directory.

\begin{verbatim}
  --t100
\end{verbatim}
This flag indicates that the job should use the alternative "T100"
timestepping options for the model (i.e. 100 timesteps per year for
the default model resolution instead of 96).

\begin{verbatim}
  -j JOB_DIR     --job-dir=JOB_DIR
\end{verbatim}
It can sometimes be useful to put GENIE jobs somewhere other than
\texttt{~/cgenie-jobs}.  This flag allows an alternative job directory
to be specified.

\begin{verbatim}
  -v MODEL_VERSION    --model-version=MODEL_VERSION
\end{verbatim}
Normally, \texttt{new-job} will generate a job set up to use the
default model version which was selected when the
\texttt{setup-cgenie} script was run.  This flag allows for a
different model version to be selected.

\subsection*{Examples}

\begin{verbatim}
  ./new-job -b cgenie.eb_go_gs_ac_bg.p0650e.NONE ...
            ... -u LABS/LAB_0.snowball snowball 10
\end{verbatim}
This configures the first example job in the workshop handout.  After
running this invocation of \texttt{new-job}, a new
\texttt{~/cgenie-jobs/snowball} job directory will have been created
from which the job can be executed.

\begin{verbatim}
  ./new-job -b cgenie.eb_go_gs_ac_bg.p0650e.NONE ...
            ... -u LABS/LAB_0.snowball -r snowball snowball2 10
\end{verbatim}
This invocation of \texttt{new-job} sets up a new \texttt{snowball2}
job that restarts from the end of the \texttt{snowball} job to run for
an additional 10 years.

%----------------------------------------------------------------------
\section{Running jobs}

Once a job has been set up using the \texttt{new-job} script, it can
be run from the newly created job directory using a ``\texttt{go}''
script.  Configuring and running a job is as simple as:
\begin{verbatim}
  cd ~/cgenie
  ./new-job -b cgenie.eb_go_gs_ac_bg.p0650e.NONE ...
            ... -u LABS/LAB_0.snowball snowball 10
  cd ~/cgenie-jobs/snowball
  ./go run
\end{verbatim}
The \texttt{go} script has three main options and two advanced
options.  The basic options are:
\begin{description}
  \item[\texttt{./go clean}]{Remove model output and model executables
    and compiled object files for the current job setup.}
  \item[\texttt{./go build}]{Compile the required version of the model
    to run this job -- this depends on a number of things, including
    the selected model resolution, but the build system ensures that
    model executables are not recompiled unnecessarily.}
  \item[\texttt{./go run}]{Compile the model (if necessary) and run
    the current job.}
\end{description}
Both the \texttt{build} and \texttt{run} commands can also take a
``build type'' argument for building debug or profiling versions of
the model.  For more information about this and about how the build
system maintains fresh executables of selected versions of the model,
see Section~\ref{sec:genie-devs}.  Also see that section for the two
``advanced'' options to the \texttt{go} script, which are used to
select alternative ``platforms'' for a machine -- in the normal case,
the build system will select the appropriate compilers and flags based
on the machine on which the model is being run (assuming that a
platform definition has been set up for the machine), but sometimes it
may be desirable to select between different compilers on the same
machine, for which a \texttt{set-platform} option is provided by the
\texttt{go} script.

%----------------------------------------------------------------------
\section{Managing configuration files}

Configuration files are all kept in \texttt{~/cgenie-data}, base
configurations in the \texttt{base-configs} directory and user
configurations in \texttt{user-configs}.  All of this configuration
data is held in a Git repository on GitHub, so if you want to add user
or base configurations to share with other users, ask someone about
how to set yourself up to use GitHub.

%----------------------------------------------------------------------
\section{Managing tests}

It is possible to save job configurations and results as test jobs
with ``known good'' data.  This has two main uses -- first, for
testing a GENIE installation to make sure that it's working; second,
to test that changes to the model don't inadvertently affect
simulation results.  The second application is of more interest for
people changing the GENIE model code, but it can still be useful to
save jobs as tests.

The \texttt{tests} script in \texttt{~/cgenie} is used to manage and
run test jobs.  To list the available tests, do
\begin{verbatim}
  ./tests list
\end{verbatim}
and to run an individual test or a set of tests, do
\begin{verbatim}
  ./tests run <test>
\end{verbatim}
where \texttt{<test>} is either a single test name (e.g.
\texttt{basic/biogem}), a set of tests (e.g. \texttt{basic}) or
\texttt{ALL}, which runs \emph{all} available tests.  The tests are
run as normal GENIE jobs in a subdirectory of \texttt{~/cgenie-jobs}
with a name of the form \texttt{test-YYYYMMDD-HHMMSS} based on the
current date and time.  As well as full test job output, build and run
logs, a \texttt{test.log} file is produced in this test directory,
plus a \texttt{summary.txt} file giving a simple pass/fail indication
for each test.

An existing job can be added as a test using a command like
\begin{verbatim}
  ./tests add <job-name>
\end{verbatim}
where \texttt{<job-name>} is the name of an existing job in
\texttt{~/cgenie-jobs}.  Note that you need to run the job before you
can add it as a test!  The test script will ask you which output files
you want to use for comparison for each model component -- there are
sensible defaults in most cases, but you can select individual files
too if you prefer.

There are two other features of the test addition command that can be
useful.  First, it's possible to give the test a different name than
the job it's made from -- for example
\begin{verbatim}
  ./tests add hosing/test-1=hosing-experiment-1
\end{verbatim}
adds a test called \texttt{hosing/test-1} based on the
\texttt{hosing-experiment-1} job.  Second, it's possible to say that a
new test should be restarted from the output of an existing test.
Normally, if a test is created from a job that requires restart files,
the restart files are just copied from the job into the new test.
Sometimes though, it can be of interest to run the job that generated
the restart data, then immediately run a test starting from the output
of the first test.  This can be done using something like this:
\begin{verbatim}
  ./tests add foo/test-1=job-1
  ./tests add foo/test-2=job-2 -r foo/test-1
\end{verbatim}
This indicates that \texttt{foo/test-1} is a ``normal'' test, while
\texttt{foo/test-2} is a test that depends on \texttt{foo/test-1} for
its restart data.  When you run a test that depends on another for
restart data, the test script deals with making sure that the restart
test is run before the test that depends on it.  So, for example, you
can just say
\begin{verbatim}
  ./tests run foo/test-2
\end{verbatim}
and the test script will figure out that it needs to run
\texttt{foo/test-1} first in order to generate restart data for
\texttt{foo/test-2}.

%----------------------------------------------------------------------
\section{Managing model versions}

For most users, it makes sense to run jobs using the most recent
available version of the GENIE model code.  This is the option chosen
by default when the model is initially set up.  However, it can
sometimes be useful to run jobs with earlier model versions (or with a
development version of the model -- see the next section).  The GENIE
configuration and build system provides a simple mechanism to permit
this, hiding most of the (rather complex) details of managing multiple
model versions from users.

Model versions are indicated by Git ``tags''.  In order to see a list
of available model versions, use the following command in the
\texttt{~/cgenie} directory:
\begin{verbatim}
  git tag -l
\end{verbatim}
To configure a job to use a different model version from the default,
simply add a \texttt{-v} flag to \texttt{new-job} specifying the model
version you want to use.  For example, to configure a job to use the
\texttt{cupcake-1.0} version of the model, use something like the
following command:
\begin{verbatim}
  ./new-job -b cgenie.eb_go_gs_ac_bg.p0650e.NONE ...
            ... -u LABS/LAB_0.snowball snowball 10 ...
            ... -v cupcake-1.0
\end{verbatim}
Within a job directory, you can see what model version the job was
configured with by looking at the contents of the
\texttt{config/model-version} file -- in non-development cases, this
will just contain the Git tag of the model version.


%======================================================================
\chapter{For GENIE developers}
\label{sec:genie-devs}

For developers of GENIE, there are a few extra things to know beyond
what's needed to run the model.

%----------------------------------------------------------------------
\section{Installation and setup for development}

Installation and setup for developers goes almost the same as for
users, except that when the \texttt{setup-cgenie} script asks for the
default model version to use, you should answer
``\texttt{DEVELOPMENT}''.  This causes model executables for all jobs
to be built by default from the source code currently in
\texttt{~/cgenie/src}, rather than from a specified past model
version.  In this way, you can make changes to the model source code
under \texttt{~/cgenie/src} and doing a \texttt{./go run} in a job
directory will trigger a build and execution of the model based on the
changed code.  (For a simpler way to check for successful model
compilation, see below in Section~\ref{sec:build-system}.)

%----------------------------------------------------------------------
\section{Model source organisation}

The model source lives in \texttt{~/cgenie/src}, with one subdirectory
for each model component (\texttt{embm}, \texttt{biogem}, etc.), the
main \texttt{genie.f90} program, plus a couple of extra subdirectories
for utility routines.  All of the code is Fortran 90 and all source
files accordingly have a \texttt{.f90} extension.

As well as Fortran source files, the per-module and base \texttt{src}
directories also contain default namelist definitions for each module
and ``exceptions'' files giving mappings from ``old'' to ``new''
parameter names that appear in GENIE configuration files and
namelists.  All of these things are used by the model configuration
system to construct valid namelists from user-specified configuration
files.

The other thing you'll see in the source directories are files called
\texttt{SConscript}.  These are the scripts that control the tool used
to manage model compilation, described in the next section.

%----------------------------------------------------------------------
\section{Build system}
\label{sec:build-system}

The build system uses SCons, a ``Make replacement'' written in
Python.  You don't need to worry about installing SCons yourself since
a suitable local version is included in the \texttt{cgenie}
repository.  The rest of the build system is also written in Python
and lives in \texttt{~/cgenie/scripts}.  In the normal course of
things, it shouldn't be necessary for GENIE model developers to touch
this stuff -- it should just work.  (There are cases where this isn't
quite true, mostly to do with major changes in the layout or naming of
model input and output files or model parameters, but the scripts have
been written as ``defensively'' as possible, so there shouldn't be
\emph{too} many problems.)

Normally the model is compiled and executed from the \texttt{go}
script in a job directory.  This deals with making sure that the
correct model version is built from the correct sources (taking
account of the model version requested) and again, should just work.

The scripts perform builds in directories under
\texttt{~/cgenie-jobs/MODELS} -- after running a few jobs, you'll see
one directory under there for each model version you've used.  The
directories all ultimately have the form
\begin{verbatim}
  ~/cgenie-jobs/MODELS/<version>/<platform>/<hash>/<build-type>
\end{verbatim}
where the different components have the following meanings:
\begin{description}
  \item[\texttt{<version>}]{The model version for this build.  If you
    run jobs with a non-\texttt{DEVELOPMENT} model version, you'll
    also see a directory called \texttt{~/cgenie-jobs/MODELS/REPOS}
    holding repository copies at fixed versions -- when a build is
    required for a non-\texttt{DEVELOPMENT} model version, the source
    code is accessed from one of these \texttt{REPOS} directories
    rather than from \texttt{~/cgenie/src}.}
  \item[\texttt{<platform>}]{A ``platform'' is basically a combination
    of a machine name or type and a compiler/NetCDF directory
    combination -- see Section~\ref{sec:platforms}.  Having this level
    in the directory structure prevents surprises with home
    directories NFS-mounted across machines with different compiler or
    NetCDF path requirements.}
  \item[\texttt{<hash>}]{One rather undesirable aspect of the current
    GENIE setup is that a number of array sizes (mostly coordinate
    dimension sizes) are compiled into the model as preprocessor
    constants.  This means that jobs using different model grid sizes
    must use different executables compiled with different
    preprocessor flags.  In order to manage this problem in a coherent
    way, the preprocessor definitions are collected into a canonical
    representation and a SHA1 hash is calculated.  This provides a
    unique representation of each model coordinate setup that can be
    used to segregate model builds.}
  \item[\texttt{<build-type>}]{Finally, we distinguish between
    \texttt{ship} (optimised), \texttt{debug} and \texttt{profile}
    builds of the model.}
\end{description}
As an example, here's a path to a build directory on my machine
(called \texttt{seneca}), for an optimised (\texttt{ship}) executable
for a job with preprocessor definitions \texttt{GENIENX=36},
\texttt{GENIENY=36}:
\begin{verbatim}
  /home/iross/cgenie-jobs/MODELS/DEVELOPMENT/seneca/...
            ...50b3ce7f3162a0f783e4424c9a294de0061e0cdc/ship
\end{verbatim}

The benefit of this arrangement is that it perfectly segregates all
the aspects of model configuration, version or build environment that
might impact the determination of what source files need to be
recompiled at any time.  Whenever you type \texttt{./go run} or
\texttt{./go build} in a job directory, only the out of date source
files for the exact model version and configuration you need are
recompiled, and the object files and executables for that exact
configuration are kept seperate from all other model configurations.
This avoids any confusing problems with stale files.

Although this is really pretty neat for managing model executables for
running jobs, it's not very convenient for day-to-day development,
where you often just want to check that the model compiles.  In order
to make this convenient, the SCons scripts are set up so that it's
possible to just run \texttt{scons} in the \texttt{~/cgenie} directory
and have \emph{a} version of the model built right there (actually
into a \texttt{~/cgenie/build} directory).  What this means is that
you can set up your editor compilation command (if you use Emacs, the
thing that gets run when you type \texttt{C-c m}) to be
``\texttt{scons -C ~/cgenie}'' and all your usual compiler error
message chasing commands will work just right.  (The \texttt{-C}
argument to the \texttt{scons} program is the same as that for
\texttt{make}: it tells SCons to change to a particular directory
before running.)  If you don't have SCons installed on your machine,
you can just run the local version in
\texttt{~/cgenie/scripts/scons/scons.py}, which should just work.

%----------------------------------------------------------------------
\section{Primary workflow}

Given the above description, here's the primary workflow I'd recommend
for working on GENIE development (the things to do with Git are
described in more detail in Section~\ref{sec:git-practices}):
\begin{enumerate}
  \item{Do \texttt{git checkout -b <new-branch-name>} to create a Git
    topic branch to work on.}
  \item{Edit files under \texttt{~/cgenie/src}.}
  \item{Test model build by running \texttt{scons -C ~/cgenie}.}
  \item{Set up test jobs using \texttt{new-job}.}
  \item{Run jobs in their job directories using \texttt{./go run}
    (uses development code).}
  \item{Can also run tests, also using development code by default.}
  \item{Build debug and profiling executables using \texttt{./go build
      debug} or \texttt{./go build profile} in the job directory.
    Debug by running \texttt{gdb} or equivalent on the
    \texttt{genie-debug.exe} executable directly in the job
    directory.}
  \item{Commit good changes and push to GitHub -- this triggers
    testing on the Travis continuous integration system.  \emph{The
      Travis stuff isn't set up yet, but it will be!}}
  \item{If the commit passed Travis testing, make a GitHub pull
    request to have your changes incorporated into the main GENIE
    repository.}
\end{enumerate}

%----------------------------------------------------------------------
\section{Model input parameter handling}

The GENIE executable reads a number of Fortran namelists to pick up
model parameters.  These namelists all have names like
\texttt{data\_EMBM}, \texttt{data\_BIOGEM}, etc., live in the top
level of each job directory and are produced by the \texttt{new-job}
script by combining values in the base and user configuration files
and default namelists for each module.

The \texttt{new-job} script is written in a way that should avoid
problems with old, no longer used, parameter names appearing in
configuration files and with configuration files that do not specify
values for newly introduced parameter names.  The way that this works
is that the default namelists for each module
(e.g. \texttt{~/cgenie/src/embm/embm-defaults.nml}) contain a default
value for every parameter appearing in the relevant namelist in the
Fortran code.  The \emph{new-job} script then ignores any parameters
appearing in configuration files whose names do not occur in the
default namelist and uses default values for any parameters in the
default namelist that do not appear in the configuration
files\footnote{Slight lie: there are some parameters that are set via
  other mechanisms than the configuration files, notably parameters to
  do with timestepping and restart files.}.  This means that if you
introduce a new parameter or remove an existing parameter from a
namelist in the Fortran code, you \emph{must} also modify the default
namelist file.

%----------------------------------------------------------------------
\section{Recommended Git working practices}
\label{sec:git-practices}

There are lots of ways of working with Git and GitHub.  Here are some
recommendations based on having used Git quite a bit for open source
work:
\begin{description}
  \item[Branches are cheap.  Use them!]{Branches work differently in
    Git than in older version control systems.  A branch is just a
    named state of the repository: switching from one branch to
    another is very quick and creating new branches costs almost
    nothing.  This enables a different workflow than in a system like
    Subversion, where branching is a more expensive operation.  When
    using Git, you can create branches for \emph{everything}.  If you
    need to make a few small changes to fix a bug, create a branch to
    do it and merge the changes into the main branch when you're done.
    If you're working on some more extensive changes, do them on a
    seperate branch and periodically pull changes to the main branch
    from the repository and merge them into your working branch so
    that you don't diverge too much from the main repository (ending
    up with a huge and complicated merge to do when you've finished).
    If you work this way, then it's very easy to have a number of
    tasks going on in parallel without confusion (because the work for
    each task is on a seperate branch), and you can always switch back
    to the main branch and make another branch if you need to start
    work on something else.  This ``one repository, many branches''
    style of working is very powerful.}
  \item[Repository forks and pull requests]{Instead of just cloning
    the \texttt{cgenie} repository and pushing changes directly to it
    (which is only possible if you're one of the owners of the
    repository), create a personal \emph{fork} of the repository on
    GitHub and do your work in that.  You set up the main repository
    as an \emph{upstream remote} of your personal fork, which allows
    you easily to pull and changes from the main repository into your
    fork.  And if you do development on branches as suggested above,
    once you're ready to submit your work for merging into the main
    repository, you can use GitHub to create a \emph{pull request} for
    your topic branch.  This is an extremely convenient way of
    communicating changes to the maintainer of the main repository.
    The maintainer gets an email saying that there's a pull request,
    the Travis continuous integration tests are run on the pull
    request branch, and the GitHub user interface provides immediate
    feedback about whether the pull request is safe to merge.  The
    maintainer can then merge your work into the main repository by
    pressing a single button.}
  \item[Issue tracker]{GitHub incorporates an \emph{issue tracker} for
    each repository.  The is really convenient for keeping track of
    bugs that need to be fixed or enhancements that people want to
    have implemented.  Each issue has a discussion thread associated
    with it so developers and maintainers can talk back and forth
    about what to do.  There are also various handy facilities for
    organising and sorting issues.  (Pull requests have the same sort
    of discussion threads, which can be useful if the work someone
    submits isn't quite right!)}
  \item[Releases as tags]{When the maintainer of a repository wants to
    create a new official release of a package, they can just create a
    Git tag on the repository.  The GENIE configuration and build
    system uses these tags to identify the available model versions,
    and they can also be listed and browsed in the GitHub interface.
    (In Git, a ``tag'' is just a name for a particular version of the
    repository so, like branches, they're very lightweight.)}
\end{description}

%----------------------------------------------------------------------
\section{Platform files}
\label{sec:platforms}

The build system has a simple facility for managing compiler paths and
options and the location of NetCDF libraries for building GENIE on
different platforms.  This mechanism is based on ``platform files'' in
this directory \texttt{~/cgenie/platforms}, each of which is named
after the hostname of the machine it's for, or the host and compiler
combination (e.g. for my machine, \texttt{seneca} or something like
\texttt{seneca-gfortran} or \texttt{seneca-ifort}).  (There is also a
default \texttt{LINUX} platform file that uses the GNU Fortran
compiler and tries to find NetCDF libraries in some ``conventional''
places, but that's really not all that likely to work in most
cases...)

Platform files are read when the \texttt{go} script for a job needs to
build the GENIE executable.  By default, the platform file named after
the machine name is used.  If you want to use an alternative compiler,
use the \texttt{set-platform} and \texttt{clear-platform} options to
the \texttt{go} script.  For example, on \texttt{seneca}, to switch to
using the Intel Fortran compiler to run a job, do the following:
\begin{verbatim}
  ./go set-platform seneca-ifort
  ./go run
\end{verbatim}
and to switch back to using the default for the machine, do:
\begin{verbatim}
  ./go clear-platform
  ./go run
\end{verbatim}
Platform files are just Python scripts that set up a few variables.
Each platform script must provide definitions for the following names:
\begin{description}
  \item[\texttt{f90}]{Fortran 90 compiler and flag setup.  Contains
    the fields:
    \begin{description}
      \item[\texttt{compiler}]{The name of the compiler executable.}
      \item[\texttt{baseflags}]{A list of compiler flags to be used
        for all compilations.}
      \item[\texttt{debug}]{A list of debug flags (used for build type
        \texttt{debug}).}
      \item[\texttt{ship}]{A list of optimisation flags (build type
        \texttt{ship}).}
      \item[\texttt{profile}]{A list of profiling flags (for build type
        \texttt{profile}).}
      \item[\texttt{profile\_link}]{A list of flags to be used for
        linking builds of type \texttt{profile} -- many compilers
        require extra flags to be given at link time for profiling
        builds.}
      \item[\texttt{bounds}]{Flags to enable array bounds checking
        (build type \texttt{bounds}).}
      \item[\texttt{include}]{\emph{Not yet used.}}
      \item[\texttt{module\_dir}]{Flag used by the Fortran 90 compiler
        to specify the location to write module files.}
      \item[\texttt{define}]{Flag used to define preprocessor
        constants.}
    \end{description}}
  \item[\texttt{netcdf}]{Options for finding and using NetCDF
    libraries.  Contains the fields:
    \begin{description}
      \item[\texttt{base}]{The base directory of the NetCDF
        installation.  Should contain an \texttt{include} subdirectory
        containing a Fortran 90 \texttt{netcdf.mod} module file
        suitable for the compiler being used and a \texttt{lib}
        subdirectory containing the NetCDF libraries.}
      \item[\texttt{libs}]{A list of NetCDF library names to link the
        GENIE executable with.  Recent NetCDF installations split the
        Fortran 90 library from the C library, so that one needs to
        link with both \texttt{libnetcdf.a} and \texttt{libnetcdff.a}
        -- in this case, this field should have the value
        \texttt{['netcdf', 'netcdff']}; otherwise it should just be
        \texttt{['netcdf']}.}
    \end{description}}
\end{description}

Porting GENIE to a new platform should require little more than making
a new platform file: just copy an existing one, ideally one that uses
the same compiler, and edit the locations of the NetCDF libraries.
(The platform file should live in \texttt{~/cgenie/platforms} and its
name should be whatever is returned from the Linux \texttt{hostname}
command.)  It you want to use different compilers on the same
platform, just create multiple platform files called
\texttt{<hostname>-<compiler>} -- you'll probably also want to have a
default platform with just the hostname so that you can do builds
without setting the platform explicitly.  If you want to do builds
directly in the \texttt{~/cgenie} directory using a non-default
compiler, you can either just move the platform files around in
\texttt{~/cgenie/platforms} so that the default file for your platform
uses the compiler you want, or (not really recommended) you can create
a file called \texttt{~/cgenie/config/platform-name} containing the
name of the platform you want to use\footnote{This suborns the
  platform selection method used in builds as run from the \texttt{go}
  script, and the reason it's not really recommended is that it will
  screw things up if your home directory if NFS-mounted and you try to
  do builds on a machine different than the one for which you've made
  the \texttt{platform-name} file.}.


%======================================================================
\chapter{For infrastructure developers}

I'm not going to write too much here -- the best way to understand
what's going on in the configuration and build scripts it to read
through the \texttt{new-job} and \texttt{go} scripts and see what they
do.  However, there are some non-obvious things that deserve a bit of
extra explanation.

\section{Python installation and shell script wrappers}

All the configuration and build scripts require Python 2.7.9 (they
might work with slightly earlier versions, but I'm being
conservative).  The \texttt{~/cgenie/scripts/find\_python} shell
script is used to figure out where a suitable Python version lives (if
it's installed globally, it will be called either \texttt{python} or
\texttt{python2} and you can check the version by doing \texttt{python
  -V}).

If no suitable version of Python is installed globally, a local
installation of Python 2.7.9 is performed.  This should work on more
or less any Linux platform -- you really do just need a working C
compiler and basic libraries.  In more or less all cases, this
installation step should happen the first time that the user runs the
\texttt{setup-cgenie} script and after that, all of the main
configuration and build scripts will pick up the right version of
Python.

The main Python scripts all live in \texttt{~/cgenie/scripts}.  The
\texttt{setup-cgenie}, \texttt{new-job}, \texttt{go} and
\texttt{tests} programs are all shell scripts that deal with making
sure that the corresponding Python scripts are invoked using the right
version of Python.

This may look a little baroque, but it's a very robust way of
insulating GENIE from problems related to old versions of Python on
poorly maintained platforms.  It seems to work well.

\section{Model versions, repositories and development code}

The whole story with the \texttt{~/cgenie-jobs/MODELS} directory
hierarchy for model builds is a little complicated, but it solves a
number of related problems:
\begin{enumerate}
  \item{How can users configure run jobs with different model versions
    in an easy way?}
  \item{How can developers build and run jobs from their development
    source tree at the same time as being able to run jobs with
    specific model versions for comparison?}
  \item{How can job configuration information be kept seperate from
    model source and executable code while maintaining reproducibility
    of jobs?}
  \item{How can rebuilds be minimised while segregating object and
    executable files from incompatible model builds?}
\end{enumerate}

When a specific model version is required for a job, the
\texttt{cgenie} source repository is unpacked into a directory under
\texttt{~/cgenie-jobs/MODELS/REPOS} at exactly that version and the
source tree and configuration scripts for that model version are used
-- this means that even if your main \texttt{~/cgenie} directory, for
example, is at version \texttt{cupcake-3.5}, if you run the
\texttt{new-job} script telling it to use version \texttt{cupcake-1.0}
for the new job, \emph{all of the configuration and build steps} for
the new job will be done with the \texttt{cupcake-1.0} versions of the
model and configuration scripts.  This allows for perfect
reproducibility of jobs between model versions.

The same principle of segregation is applied to platform dependencies
and ``job hashing''.  The best way to explain this is with an
example.  Suppose that you're working on one of the modules of GENIE
making science changes that potentially have platform-dependent
effects and that also have effects that depend on model resolution.
You've set up a bunch of test jobs with different model resolutions
and other characteristics that you use to make sure you don't break
things as you make changes.  If you change a single Fortran file and
want to rerun all your tests, how many files need to be recompiled?
If you make a new test with a different model resolution, how many
Fortran files need to be recompiled?  In the first case, the answer
should be ``one, or possibly a few if the file I changed is
\texttt{USE}d in other files''.  In the second case, the answer should
be ``all of them, unless I've already built a model for that
resolution recently''.

By keeping model builds for different model versions, platforms and
model resolutions completely seperate, we can make sure that only
\emph{exactly} the files that need to be recompiled \emph{are}
recompiled in every case.  You should never need to ``clean'' all of
the build directories for a model (although the \texttt{go} script
provides that capability if you really want it) since we maintain
exact dependency information at all times.

All of this relies on some careful setup of the SCons
\texttt{SConstruct} and \texttt{SConscript} files, explained a bit
more below.

\section{Job hashing}

Because of the way that coordinate sizes for arrays are currently
defined in GENIE (via preprocessor constants), different executables
are needed for different grid sizes (and for different numbers of
tracers).  A simple hashing approach based on the coordinate
definitions in each job is used to keep this organised.

Each job directory has a \texttt{job.py} file in its \texttt{config}
subdirectory.  The \texttt{job.py} file defines the preprocessor
definitions that are needed to build the model version for the job --
this file is read as part of the build process to set up the required
preprocessor definitions.

When the \texttt{go} script needs to work out which directory to use
under \texttt{~/cgenie-jobs/MODELS} to use for the model for the
current job, it reads the \texttt{jobs.py} file and turns the
coordinate definitions into a canonical form (basically just by
stripping whitespace and line endings, and sorting the variable names,
producing a string something like
``\texttt{'GENIENX':36,'GENIENY':36}'').  The SHA1 hash of this string
is then computed (giving a long string of hex digits and this can then
be used as a unique identifier of the job coordinate definition.

\section{SConstruct organisation}

The way that the GENIE build system using SConstruct is a little bit
complicated, mostly in order to manage the segregation of different
model builds as described above.  The main
\texttt{~/cgenie/SConstruct} SCons file depends on reading some
supporting files from the model build directory (which is either one
of the directories under \texttt{~/cgenie-jobs/MODELS} or, for the
degenerate case of testing model compilation, \texttt{~/cgenie}).

Those supporting files are the \texttt{job.py} file that defines the
model coordinate sizes, described in the previous section, and a file
called \texttt{version.py} that tells the \texttt{SConstruct} script
where to find the model source code (\texttt{~/cgenie/src} for
development builds, or a directory under
\texttt{~/cgenie-jobs/MODELS/REPOS} for builds using a specified model
version) and build scripts and the type of build to perform
(e.g. \texttt{ship} or \texttt{debug}).

If you look in one of the model build directories under
\texttt{~/cgenie-jobs/MODELS}, you'll find that that's more or less
all there is there, apart from the \texttt{SConstruct} file, the
\texttt{build} directory where compilation output goes, a
\texttt{build.log} file and the \texttt{genie.exe} executable output.
Setting things up this way means that model source code and build
results are always kept seperate and there should never be any
confusion.

That said, here's a \textbf{warning}: the \texttt{SConstruct} file
uses an SCons feature called ``variant directory builds'' to pull off
the feat of building different model versions in all sorts of
different places from the same source tree.  The interaction between
this feature and SCons's automatic scanning of Fortran 90 inter-module
dependencies is very very delicate -- if you change this in \emph{any}
way, I can almost guarantee that you will break it!

\section{Data file setup for jobs}

This is possibly the nastiest and most uncertain part of the model
configuration scripts.  Because of the way that the namelist
parameters are defined for some of the GENIE model components, it's
not possible to determine exactly which model input files (from
\texttt{~/cgenie/data} or from \texttt{~/cgenie-data/forcings}) are
required to run a particular job.  That's a bit of a problem, since
the idea is that each job directory under \texttt{~/cgenie-jobs}
should be self-contained so that you could tar them up and send them
to someone else for them to duplicate the job you were running, or for
archiving purposes.  This is also important for producing
self-contained test cases.  To make this possible, each job directory
has an \texttt{input} subdirectory where all the input files required
to run the job live.  Getting the required files into that directory
requires a little bit of ingenuity.

The \texttt{copy\_data\_files} routine (in
\texttt{~/cgenie/scripts/config\_utils.py}) uses some simple
heuristics to figure out what input files might be needed.  For each
GENIE component, it extracts a list of candidate filenames from the
namelist for that component -- basically all string-link parameters
that aren't obviously not filenames.  It then does three things in
order, eliminating candidates that are successfully located before
going on to the next step -- in each case, if a candidate is matched,
the match is copied to the job's \texttt{input} directory:
\begin{enumerate}
  \item{Look in the \texttt{~/cgenie/data} subdirectory for the
    relevant model component for an exact match to the candidate.}
  \item{Look in \texttt{~/cgenie-data/forcings} for an exact match to
    the candidate.}
  \item{Look in the \texttt{~/cgenie/data} subdirectory for the
    relevant model component for partial matches to the candidate,
    i.e. files whose name contain the string we're looking for but
    aren't an exact match.}
\end{enumerate}

The end result of this is that the job's \texttt{input} directory ends
up containing all the file's that are needed to run the job plus
(often) some extraneous files that just happen to have similar names
to option values used in the component namelist.

Now, this is pretty ugly and prone to breakage, and is close to the
top of my list of candidates for ``a better way''.  The solution I
have in mind requires some significant changes to the way that model
configurations are set up though, and should probably wait until we
have a GUI for configuring GENIE jobs.  In the meantime, this end of
things just needs to be tested carefully...


%======================================================================
\appendix
\chapter{(Some) teaching labs updated for \texttt{cupcake}}

A few configuration and data files needed some fixups for these things
to work -- these changes have all been committed to the
\texttt{cgenie-data} repository.

\section{Session \#0000}

\subsection*{Section 1}

\begin{verbatim}
  git clone https://github.com/genie-model/cgenie.git
  cd ~/cgenie
  ./setup-cgenie
  ./tests run basic
\end{verbatim}

\subsection*{Sections 2--6}

\begin{verbatim}
  ./new-job -b cgenie.eb_go_gs_ac_bg.p0650e.NONE -u LABS/LAB_0.snowball ...
            ... LAB_0.snowball 10
  cd ~/cgenie-jobs/LAB_0.snowball
  ./go run
  cd output/biogem
  cat biogem_series_ocn_temp.res
\end{verbatim}

\subsection*{Section 7}

\begin{verbatim}
  cd ~/cgenie_output
  wget http://www.seao2.info/cgenie/labs/UoB.2013/...
            ...130328.p0650e.LiCa.OHM10.SPIN0.tar.gz
  tar xzf 130328.p0650e.LiCa.OHM10.SPIN0.tar.gz
  cd ~/cgenie
  ./new-job -b cgenie.eb_go_gs_ac_bg.p0650e.NONE -u LABS/LAB_0.snowball ...
            ... LAB_0.snowball-from-restart 100 ...
            ... -r 130328.p0650e.LiCa.OHM10.SPIN0 --old-restart
  cd ~/cgenie-jobs/LAB_0.snowball-from-restart
  ./go run
\end{verbatim}

\subsection*{Section 8}

\begin{verbatim}
  cd ~/cgenie-data/user-configs
  cp LAB_0.snowball LAB_0.snowball-experiment
\end{verbatim}

Then edit the \texttt{LAB\_0.snowball-experiment} configuraton file to
change the \texttt{ea\_radfor\_scl\_co2} variable (to 10.0, say).

\begin{verbatim}
  cd ~/cgenie
  ./new-job -b cgenie.eb_go_gs_ac_bg.p0650e.NONE ...
            ... -u LABS/LAB_0.snowball-experiment ...
            ... LAB_0.snowball-experiment-1 100
  cd ~/cgenie-jobs/LAB_0.snowball-experiment-1
  ./go run
\end{verbatim}

It's easy to make another job to extend this simulation -- just
restart from the end of the last job:
\begin{verbatim}
  cd ~/cgenie
  ./new-job -b cgenie.eb_go_gs_ac_bg.p0650e.NONE ...
            ... -u LABS/LAB_0.snowball-experiment ...
            ... LAB_0.snowball-experiment-1-extend 100 ...
            ... -r LAB_0.snowball-experiment-1
  cd ~/cgenie-jobs/LAB_0.snowball-experiment-1-extend
  ./go run
\end{verbatim}
You can do this indefinitely...

\section{Session \#0001}

\subsection*{Section 1.3}

\begin{verbatim}
  cd ~/cgenie_output
  wget http://www.seao2.info/cgenie/labs/UoB.2013/...
            ...EXAMPLE.worjh2.PO4Fe.SPIN.tar.gz
  tar xzf EXAMPLE.worjh2.PO4Fe.SPIN.tar.gz
  cd ~/cgenie-data/user-configs/LABS
  cp LAB_1.colorinjection LAB_1.colorinjection-experiment
\end{verbatim}

Edit the \texttt{LAB\_1.colorinjection-experiment} file as described
in the lab script.

\begin{verbatim}
  ./new-job -b cgenie.eb_go_gs_ac_bg.worjh2.rb ...
            ... -u LABS/LAB_1.colorinjection-experiment ...
            ... LAB_1.colorinjection 20 ...
            ... -r EXAMPLE.worjh2.PO4Fe.SPIN --old-restart
  cd ~/cgenie-jobs/LAB_1.colorinjection
  ./go run
\end{verbatim}

\subsection*{Section 1.5}

\begin{verbatim}
  ./new-job -b cgenie.eb_go_gs_ac_bg.worjh2.rb ...
            ... -u LABS/LAB_1.hosing LAB_1.hosing 20 ...
            ... -r EXAMPLE.worjh2.PO4Fe.SPIN --old-restart
  cd ~/cgenie-jobs/LAB_1.hosing
  ./go run
\end{verbatim}

\section{Session \#0100}

\subsection*{Section 1.1}

\begin{verbatim}
  ./new-job -O -b cgenie.eb_go_gs_ac_bg.worjh2.BASEFe ...
            ... -u LABS/LAB_2.CO2emissions LAB_2.CO2emissions 20 ...
            ...  -r EXAMPLE.worjh2.PO4Fe.SPIN --old-restart
  cd ~/cgenie-jobs/LAB_2.CO2emissions
  ./go run
\end{verbatim}

\subsection*{Section 1.2}

\begin{verbatim}
  ./new-job -O -b cgenie.eb_go_gs_ac_bg.worjh2.BASEFe ...
            ... -u LABS/LAB_2.CONTROL LAB_2.CONTROL 20 ...
            ...  -r EXAMPLE.worjh2.PO4Fe.SPIN --old-restart
  cd ~/cgenie-jobs/LAB_2.CONTROL
  ./go run
\end{verbatim}

\end{document}
