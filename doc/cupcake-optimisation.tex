\documentclass[a4paper,10pt,article]{memoir}

\usepackage[a4paper,margin=1in]{geometry}

\usepackage[utf8]{inputenc}
\usepackage{fourier}
\usepackage{amsmath,amssymb}

\usepackage{color}
\definecolor{orange}{rgb}{0.75,0.5,0}
\definecolor{magenta}{rgb}{1,0,1}
\definecolor{cyan}{rgb}{0,1,1}
\definecolor{grey}{rgb}{0.25,0.25,0.25}
\newcommand{\outline}[1]{{\color{grey}{\scriptsize #1}}}
\newcommand{\revnote}[1]{{\color{red}\textit{\textbf{#1}}}}
\newcommand{\note}[1]{{\color{blue}\textit{\textbf{#1}}}}
\newcommand{\citenote}[1]{{\color{orange}{[\textit{\textbf{#1}}]}}}

\usepackage{xspace}

\usepackage{listings}
\usepackage{courier}
\lstset{basicstyle=\tiny\ttfamily,breaklines=true,language=Python}

\usepackage{fancyvrb}
\usepackage{url}

\usepackage{float}
\newfloat{listing}{tbp}{lop}
\floatname{listing}{Listing}

\title{Optimisation work for GENIE \texttt{cupcake}}
\author{Ian~Ross}
\date{2 April 2015}

\begin{document}

\catcode`~=11    % Make tilde a normal character (danger of weirdness...)
%\catcode`~=13    % Make tilde an active character again

\maketitle

General guidelines:
\begin{enumerate}
  \item{Rationalise coordinate dimension size variables before doing
    anything else -- all this \texttt{maxi} vs. \texttt{imax} stuff is
    just confusing and unnecessary.}
  \item{Convert arrays to \texttt{ALLOCATABLE} in small groups and do
    a little test to make sure you're not screwing things up!}
  \item{Initialise all allocated arrays to zero: the original
    fixed-size arrays are mostly defined as \texttt{SAVE} and there
    are some places that seem to rely on them being zeroed.}
  \item{Be careful about using the \texttt{maxi} and \texttt{maxj}
    coordinate dimensions before they're properly initialised!  This
    is especially possible in the various \texttt{initialise\_...}
    routines.}
  \item{Some modules haven't been fully organised as F90 modules, so
    that also needs to be done: mostly it's a matter of putting the
    initialisation, step and tear-down routines into a single module.}
\end{enumerate}

\textbf{Not yet done:}
\begin{itemize}
  \item{Tracer reorganisation}
  \item{\texttt{maxnyr} and \texttt{maxisles} in GOLDSTEIN}
  \item{Status and error checks for \texttt{ALLOCATE}s}
\end{itemize}

%======================================================================
\chapter{Shared}

\section{Tracer counts}

Tracer counts are defined in \texttt{common/gem\_cmn.f90} as
\texttt{PARAMETER}s:

\begin{verbatim}
  INTEGER,PARAMETER::n_atm = 19
  INTEGER,PARAMETER::n_ocn = 95
  INTEGER,PARAMETER::n_sed = 79
\end{verbatim}

and also in \texttt{wrappers/genie\_control.f90}:

\begin{verbatim}
  INTEGER, PARAMETER :: intrac_atm_max=19, intrac_ocn_max=95, intrac_sed_max=79
\end{verbatim}

Instead, these \emph{maximum} tracer counts should be kept as
\texttt{PARAMETER}s while the \emph{actual} tracer counts are
determined from the \texttt{atm\_select}, \texttt{ocn\_select} and
\texttt{sed\_select} arrays:

\begin{verbatim}
  INTEGER, PARAMETER :: &
    & n_atm_max = 19, n_ocn_max = 95, n_sed_max = 79

  ...

  n_atm = COUNT(atm_select)
  n_ocn = COUNT(ocn_select)
  n_sed = COUNT(sed_select)
\end{verbatim}

and index mapping arrays should be defined as:

\begin{verbatim}
  INTEGER, DIMENSION(:), ALLOCATABLE :: &
    & idx_to_atm, idx_to_ocn, idx_to_sed
  INTEGER, DIMENSION(n_atm_max) :: atm_to_idx
  INTEGER, DIMENSION(n_ocn_max) :: ocn_to_idx
  INTEGER, DIMENSION(n_sed_max) :: sed_to_idx
\end{verbatim}

one set giving the mapping from the index into tracer arrays to the
tracer ID (all the existing \texttt{io\_...} constants) and the other
giving the mapping from the tracer ID to the index into tracer arrays
(with a default value of $-1$ for unused tracers).  These things
should all be set up immediately after the main GENIE namelist is read
so that all of the array sizes are assigned before any of the
sub-modules are initialised.

Some of this sort of thing has already been partially done in
\texttt{common/gem\_util.f90} and \texttt{common/gem\_cmn.f90}.  That
stuff should be cleaned up and used for this.  There should be one
common set of tracer index maps for each tracer type (atmosphere,
ocean, sediment) across all modules.

\emph{Might it be a good idea to actually make the \texttt{io\_...}
  constants non-constant and assign the index values to them
  directly?  Depends on how these things are used elsewhere.}

%======================================================================
\chapter{\texttt{gemlite}}

\section{Source files}

\begin{tabular}{rl}
   15 & \texttt{end\_gemlite.f90} \\
   89 & \texttt{gemlite\_data.f90} \\
   95 & \texttt{gemlite\_lib.f90} \\
  113 & \texttt{cpl\_comp\_gemlite.f90} \\
  113 & \texttt{initialise\_gemlite.f90} \\
  541 & \texttt{gemlite.f90} \\
\end{tabular}

\section{Coordinate variables}

These need to be redefined \emph{not} to be \texttt{PARAMETER}s:

\begin{verbatim}
  INTEGER,PARAMETER::n_i = ilon1_ocn
  INTEGER,PARAMETER::n_j = ilat1_ocn
  INTEGER,PARAMETER::n_k = inl1_ocn
\end{verbatim}

\section{Array definitions}

Most are already \texttt{ALLOCATABLE}, except for:

\begin{verbatim}
  INTEGER,DIMENSION(n_i,n_j)::goldstein_k1
  REAL,DIMENSION(n_k)::goldstein_dz
  REAL,DIMENSION(n_k)::goldstein_dza
  REAL,DIMENSION(0:n_j)::goldstein_sv
\end{verbatim}

\section{Allocation and deallocation}

\begin{itemize}
  \item{Needed to set up \texttt{gemlite} as a proper F90 module in
    order to have calling interfaces that will work with assumed-shape
    arrays.}
  \item{Tracer counts!  (Probably like more or less all the other
    biogeochemistry modules.)}
\end{itemize}

\section{Special considerations}

The \texttt{maxisles} variable is used in the following allocations:

\begin{verbatim}
    ALLOCATE(lpisl(mpi,maxisles))   ; lpisl = 0
    ALLOCATE(ipisl(mpi,maxisles))   ; ipisl = 0
    ALLOCATE(jpisl(mpi,maxisles))   ; jpisl = 0
    ALLOCATE(npi(maxisles))         ; npi = 0
    ALLOCATE(psisl(0:maxi,0:maxj,maxisles))          ; psisl = 0.0
    ALLOCATE(ubisl(2,0:maxi+1,0:maxj,maxisles))      ; ubisl = 0.0
    ALLOCATE(erisl(maxisles,maxisles+1))             ; erisl = 0.0
    ALLOCATE(psibc(maxisles))                        ; psibc = 0.0
\end{verbatim}

All now allocated to the exact number of islands.

%======================================================================
\chapter{\texttt{goldsteinseaice}}

\section{Source files}

\begin{tabular}{rl}
   84 & \texttt{gold\_seaice\_lib.f90} \\
  355 & \texttt{gold\_seaice\_netcdf.f90} \\
  445 & \texttt{gold\_seaice\_data.f90} \\
 1027 & \texttt{gold\_seaice.f90} \\
\end{tabular}

\section{Coordinate variables}

These need to be redefined \emph{not} to be \texttt{PARAMETER}s:

\begin{verbatim}
  INTEGER, PARAMETER :: maxi = GOLDSTEINNLONS
  INTEGER, PARAMETER :: maxj = GOLDSTEINNLATS
  INTEGER, PARAMETER :: maxk = GOLDSTEINNLEVS
\end{verbatim}

\section{Array definitions}

All defined in \texttt{gold\_seaice\_lib.f90}; all need to be made
\texttt{ALLOCATABLE}:

\begin{verbatim}
  INTEGER :: k1(0:maxi+1,0:maxj+1)
  REAL, DIMENSION(0:maxj) :: s, c, sv, cv
  REAL :: ds(maxj), dsv(1:maxj-1), rds2(2:maxj-1), u(2,0:maxi,0:maxj)
  REAL, DIMENSION(0:maxj) :: rc, rcv, cv2, rc2
  REAL :: rds(maxj), rdsv(1:maxj-1)
  REAL, DIMENSION(maxj) :: asurf
  REAL, DIMENSION(2,maxi,maxj) :: varice, varice1, dtha, varicedy, variceth
  REAL, DIMENSION(maxi,maxj) :: tice, albice
  REAL, DIMENSION(2,maxi,maxj) :: haavg, dthaavg
  REAL, DIMENSION(maxi,maxj) :: ticeavg, albiceavg, fxdelavg ,fwdelavg
\end{verbatim}

as do the following from \texttt{gold\_seaice\_netcdf.f90}:

\begin{verbatim}
  REAL, DIMENSION(maxi) :: nclon1, nclon2, nclon3
  REAL, DIMENSION(maxj) :: nclat1, nclat2, nclat3
\end{verbatim}

\section{Allocation and deallocation}

Need to be careful not to use coordinate size variables in subroutine
argument definitions \emph{before they're assigned values!}  The
Fortran compiler doesn't help you to spot this, so it produces weird
run-time errors.


%======================================================================
\chapter{\texttt{ents}}

\section{Source files}

\begin{tabular}{rl}
  129 & \texttt{ents\_data.f90} \\
  195 & \texttt{ents\_lib.f90} \\
  654 & \texttt{ents\_netcdf.f90} \\
  673 & \texttt{ents\_diag.f90} \\
  802 & \texttt{ents.f90} \\
\end{tabular}

\section{Coordinate variables}

Same as in other modules:

\begin{verbatim}
  INTEGER, PARAMETER :: maxi=GOLDSTEINNLONS, maxj=GOLDSTEINNLATS
\end{verbatim}

(Also the usual \texttt{maxi}/\texttt{imax},
\texttt{maxj}/\texttt{jmax} duplication.)

\section{Array definitions}

All defined in \texttt{gold\_seaice\_lib.f90}; all need to be made
\texttt{ALLOCATABLE}:

\begin{verbatim}
  INTEGER :: ents_k1(maxi,maxj)
  REAL :: ents_lat(maxj)
  REAL, DIMENSION(maxi,maxj) :: Cveg, Csoil, fv, epsv
  REAL, DIMENSION(maxi,maxj) :: leaf, respveg, respsoil, photo
  REAL, DIMENSION(maxi,maxj) :: sphoto, srveg, sleaf, srsoil, sCveg1, &
    & sCsoil1, sfv1, sepsv1, sfx0a, sfx0o, sfxsens, sfxlw, sevap, &
    & spptn, srelh, sbcap, salbs, ssnow, sz0
  REAL :: stqld(2,maxi,maxj)
  REAL, DIMENSION(2,maxi,maxj) :: tqld, tqldavg
  REAL, DIMENSION(maxi,maxj) :: bcap, bcapavg, snowavg, z0avg, &
    albsavg, z0, evapavg, pptnavg, runavg, fvfv
  REAL :: fxavg(7,maxi,maxj)
\end{verbatim}

\section{Allocation and deallocation}

Some cases where initialisation is really needed: all the fixed-size
arrays were defined as \texttt{SAVE} so they got initialised
automatically.


%======================================================================
\chapter{\texttt{embm}}

\section{Source files}

\begin{tabular}{rl}
   250 & \texttt{embm\_lib.f90} \\
   294 & \texttt{embm\_data.f90} \\
   412 & \texttt{embm\_netcdf.f90} \\
   794 & \texttt{embm\_diag.f90} \\
  3593 & \texttt{embm.f90} \\
\end{tabular}

\section{Coordinate variables}

\begin{verbatim}
  INTEGER, PARAMETER :: maxi=GOLDSTEINNLONS, maxj=GOLDSTEINNLATS
  INTEGER, PARAMETER :: maxk=GOLDSTEINNLEVS, maxl=2
  INTEGER, PARAMETER :: maxnyr=400
  INTEGER, PARAMETER :: en_ntimes_max=2000
\end{verbatim}

Here, \texttt{maxl} isn't used, but these \texttt{maxnyr} and
\texttt{en\_ntimes\_max} things will require some attention.

\section{Array definitions}

\begin{verbatim}
  INTEGER :: k1(0:maxi+1,0:maxj+1)
  INTEGER :: ku(2,maxi,maxj), mk(maxi+1,maxj)
  REAL :: dt(maxk), ds(maxj), dsv(1:maxj-1), rds2(2:maxj-1), &
       & dz(maxk), s(0:maxj), c(0:maxj), dzu(2,maxk), &
       & tau(2,maxi,maxj), drag(2,maxi+1,maxj), dztau(2,maxi,maxj), &
       & diff(2), ec(4), sv(0:maxj)
  REAL :: cv(0:maxj), dza(maxk), dztav(2,maxi,maxj), &
       & tau0(maxi,maxj), dztav0(maxi,maxj), &
       & tau1(maxi,maxj), dztav1(maxi,maxj), tsa0(maxj)
  REAL :: rc(0:maxj), rcv(1:maxj-1), rdphi, rds(maxj), rdsv(1:maxj-1), &
       & cv2(1:maxj-1), rc2(0:maxj), rtv(maxi,maxj), rtv3(maxi,maxj), &
       & rdz(maxk), rdza(maxk)
  REAL :: us_dztau(2, maxi, maxj), us_dztav(2, maxi, maxj)
  REAL :: asurf(maxj)
  REAL :: tq(2,maxi,maxj), tq1(2,maxi,maxj), qsata(maxi,maxj), &
       & qsato(maxi,maxj), co2(maxi,maxj), ch4(maxi,maxj), n2o(maxi,maxj), &
       & varice(2,maxi,maxj), varice1(2,maxi,maxj), &
       & tqa(2,maxi,maxj), solfor(maxj,maxnyr)
  REAL :: albcl(maxi,maxj), fxsw(maxi,maxj), fxplw(maxi,maxj), &
       & fx0a(maxi,maxj), fx0o(maxi,maxj), fxsen(maxi,maxj), &
       & pmeadj(maxi,maxj), pptn(maxi,maxj), evap(maxi,maxj), &
       & usurf(maxi,maxj), fxlata(maxi,maxj), fxlato(maxi,maxj), &
       & fxlw(maxi,maxj), diffa(2,2,maxj), betam(2), betaz(2), hatmbl(2), &
       & ca(maxi,maxj), qb(maxi,maxj), qbsic(maxi,maxj)
  REAL :: fx0sic(maxi,maxj), fx0neto(maxi,maxj), fwfxneto(maxi,maxj), &
       & evapsic(maxi,maxj), tsfreez(maxi,maxj)
  REAL :: uatm(2,maxi,maxj)
  REAL :: tqavg(2,maxi,maxj), fxlatavg(maxi,maxj), fxsenavg(maxi,maxj), &
       & fxswavg(maxi,maxj), fxlwavg(maxi,maxj), fwpptavg(maxi,maxj), &
       & fwevpavg(maxi,maxj)
  REAL :: fx0avg(4,maxi,maxj), fwavg(2,maxi,maxj)
  REAL :: albo(maxj,maxnyr), palb(maxi,maxj), palbavg(maxi,maxj)
  REAL :: lice_vect(maxi,maxj,en_ntimes_max)
  REAL, DIMENSION(maxi,maxj) :: d18o_ice_thresh, d18o_orog_min, d18o_orog_grad
  REAL :: uatml(2,maxi,maxj,maxnyr)
  REAL, DIMENSION(maxi,maxj,maxnyr) :: usurfl, tncep, pncep, rhncep, atm_alb
  REAL, DIMENSION(maxi,maxj) :: chl, cel
  REAL, DIMENSION(maxi,maxj) :: q_pa, rq_pa, q_pa_avg, rq_pa_avg
  INTEGER, DIMENSION(maxi,maxj) :: iroff, jroff
\end{verbatim}

\section{Special considerations}

The \texttt{maxnyr} (400) and \texttt{en\_ntimes\_max} (2000) things
are used in the following allocations:

\begin{verbatim}
    ALLOCATE(solfor(maxj,maxnyr))
    ALLOCATE(albo(maxj,maxnyr))
    ALLOCATE(uatml(2,maxi,maxj,maxnyr))
    ALLOCATE(usurfl(maxi,maxj,maxnyr))
    ALLOCATE(tncep(maxi,maxj,maxnyr))
    ALLOCATE(pncep(maxi,maxj,maxnyr))
    ALLOCATE(rhncep(maxi,maxj,maxnyr))
    ALLOCATE(atm_alb(maxi,maxj,maxnyr))

    REAL :: orbitall_vect(en_ntimes_max,5) [local]
    ALLOCATE(orog_vect(maxi,maxj,en_ntimes_max)) ; orog_vect = 0.0
    ALLOCATE(lice_vect(maxi,maxj,en_ntimes_max)) ; lice_vect = 0.0
\end{verbatim}

Now removed and replaced with more appropriate allocations where
required.

%======================================================================
\chapter{\texttt{atchem}}

\section{Source files}

\begin{tabular}{rl}
   15 & \texttt{end\_atchem.f90} \\
   41 & \texttt{initialise\_atchem.f90} \\
   66 & \texttt{cpl\_flux\_atchem.f90} \\
  110 & \texttt{atchem\_data\_netCDF.f90} \\
  131 & \texttt{atchem\_lib.f90} \\
  146 & \texttt{cpl\_comp\_atchem.f90} \\
  149 & \texttt{atchem\_box.f90} \\
  154 & \texttt{atchem.f90} \\
  285 & \texttt{atchem\_data.f90} \\
\end{tabular}

\section{Coordinate variables}

\begin{verbatim}
  INTEGER,PARAMETER::n_i = ilon1_atm
  INTEGER,PARAMETER::n_j = ilat1_atm
  INTEGER,PARAMETER::n_phys_atm = 15
\end{verbatim}

The \texttt{n\_phys\_atm} variable will be dealt with in the
``tracers'' stuff.

\section{Array definitions}

\begin{verbatim}
  real,dimension(n_atm,n_i,n_j) :: atm
  real,dimension(n_atm,n_i,n_j) :: fatm
  real,dimension(n_phys_atm,n_i,n_j) :: phys_atm
  real,dimension(n_atm,n_i,n_j) :: atm_slabbiosphere
\end{verbatim}


%======================================================================
\chapter{\texttt{rokgem}}

\section{Source files}

\begin{tabular}{rl}
  29 & \texttt{end\_rokgem.f90} \\
  46 & \texttt{cpl\_flux\_rokatm.f90} \\
  66 & \texttt{cpl\_flux\_rokocn.f90} \\
 167 & \texttt{rokgem.f90} \\
 274 & \texttt{initialise\_rokgem.f90} \\
 405 & \texttt{rokgem\_lib.f90} \\
 453 & \texttt{rokgem\_data\_netCDF.f90} \\
 827 & \texttt{rokgem\_data.f90} \\
1737 & \texttt{rokgem\_box.f90} \\
\end{tabular}

\section{Coordinate variables}

\begin{verbatim}
  INTEGER,PARAMETER :: n_i = ilon1_rok
  INTEGER,PARAMETER :: n_j = ilat1_rok
  INTEGER,PARAMETER :: n_phys_rok = 08
  INTEGER,PARAMETER :: n_phys_ocnrok = 06
  INTEGER,PARAMETER :: n_io = ilon1_rok
  INTEGER,PARAMETER :: n_jo = ilat1_rok
  INTEGER,PARAMETER :: n_ko = inl1_ocn
\end{verbatim}

\section{Array definitions}

Some are already \texttt{ALLOCATABLE}...

\begin{verbatim}
  real,dimension(n_phys_rok,n_i,n_j)::phys_rok
  REAL,DIMENSION(n_phys_ocnrok,n_io,n_jo) :: phys_ocnrok
  INTEGER,DIMENSION(ilon1_ocn,ilat1_ocn) :: goldstein_k1
  INTEGER :: landmask(n_i,n_j)
  REAL :: runoff_drainage(n_i+2,n_j+2)
  INTEGER :: runoff_drainto(n_i,n_j,2)
  REAL :: runoff_coast(n_i,n_j)
  REAL :: total_calcium_flux(n_i,n_j)
  REAL :: total_calcium_flux_Ca(n_i,n_j)
  REAL :: total_calcium_flux_Si(n_i,n_j)
  REAL :: weather_fCaCO3_2D(n_i,n_j)
  REAL :: weather_fCaSiO3_2D(n_i,n_j)
  REAL :: orogeny(n_i,n_j)
  REAL :: regimes_calib(n_i,n_j)
  REAL :: ref_T0_2D(n_i,n_j)
  REAL :: ref_R0_2D(n_i,n_j)
  REAL :: ref_P0_2D(n_i,n_j)
  REAL :: data_T_2D(n_i,n_j)
  REAL :: data_R_2D(n_i,n_j)
  REAL :: data_P_2D(n_i,n_j)
  REAL :: calibrate_T_2D(n_i,n_j)
  REAL :: calibrate_R_2D(n_i,n_j)
  REAL :: calibrate_P_2D(n_i,n_j)
\end{verbatim}

\section{Allocation and deallocation}


%======================================================================
\chapter{Miscellaneous}

\section{Source files}

\section{Coordinate variables}

\section{Array definitions}

\section{Allocation and deallocation}


%======================================================================
\chapter{\texttt{goldstein}}

\section{Source files}

\begin{tabular}{rl}
   260 & \texttt{goldstein\_lib.f90} \\
   446 & \texttt{goldstein\_data.f90} \\
   718 & \texttt{goldstein\_netcdf.f90} \\
  1147 & \texttt{goldstein\_diag.f90} \\
  3524 & \texttt{goldstein.f90} \\
\end{tabular}

\section{Coordinate variables}

\begin{verbatim}
  INTEGER, PARAMETER :: maxi=GOLDSTEINNLONS, maxj=GOLDSTEINNLATS
  INTEGER, PARAMETER :: maxk=GOLDSTEINNLEVS, maxl=GOLDSTEINNTRACS
  INTEGER, PARAMETER :: maxnyr=720
  INTEGER, PARAMETER :: mpxi=maxi, mpxj=maxj+1
  INTEGER, PARAMETER :: maxisles=GOLDSTEINMAXISLES, mpi=2 * (maxi + maxj)
\end{verbatim}

\section{Array definitions}

\begin{verbatim}
  INTEGER :: k1(0:maxi+1,0:maxj+1), ku(2,maxi,maxj), mk(maxi+1,maxj)
  INTEGER :: ips(maxj), ipf(maxj), ias(maxj), iaf(maxj)
  INTEGER :: lpisl(mpi,maxisles), ipisl(mpi,maxisles), jpisl(mpi,maxisles)
  INTEGER :: npi(maxisles)
  REAL :: dt(maxk), ds(maxj), dsv(1:maxj-1), rds2(2:maxj-1)
  REAL :: dz(maxk), u(3,0:maxi,0:maxj,maxk), ts(maxl,0:maxi+1,0:maxj+1,0:maxk+1)
  REAL :: s(0:maxj), c(0:maxj), dzu(2,maxk), tau(2,maxi,maxj)
  REAL :: drag(2,maxi+1,maxj), dztau(2,maxi,maxj)
  REAL :: ratm(mpxi*mpxj,mpxi+1), ub(2,0:maxi+1,0:maxj)
  REAL :: rho(0:maxi+1,0:maxj+1,0:maxk), ts1(maxl,0:maxi+1,0:maxj+1,0:maxk+1)
  REAL :: sv(0:maxj)
  REAL :: cv(0:maxj), dza(maxk), dztav(2,maxi,maxj), gb(mpxi*mpxj)
  REAL :: gap(mpxi*mpxj,2*mpxi+3), cost(maxi,maxj), rh(3,0:maxi+1,0:maxj+1)
  REAL :: gbold(mpxi*mpxj), tau0(maxi,maxj), dztav0(maxi,maxj)
  REAL :: tau1(maxi,maxj), dztav1(maxi,maxj), tsa0(maxj), t0
  REAL :: fw_hosing(maxi,maxj), rhosing(maxi,maxj), zro(maxk), zw(0:maxk)
  REAL :: dzg(maxk,maxk), z2dzg(maxk,maxk), rdzg(maxk,maxk)
  REAL :: fw_anom(maxi,maxj), fw_anom_rate(maxi,maxj)
  REAL :: psi(0:maxi,0:maxj)
  REAL :: u1(3,0:maxi,0:maxj,maxk)
  REAL, DIMENSION(0:maxj) :: rc, rc2
  REAL, DIMENSION(maxi,maxj) :: rtv, rtv3
  REAL, DIMENSION(1:maxj-1) :: rcv, rdsv, cv2
  REAL :: rds(maxj), rdz(maxk), rdza(maxk)
  REAL :: bp(maxi+1,maxj,maxk), sbp(maxi+1,maxj)
  INTEGER :: icosd(maxi,maxj)
  REAL :: asurf(maxj)
  REAL :: tsavg(maxl,0:maxi+1,0:maxj+1,0:maxk+1)
  REAL :: uavg(3,0:maxi,0:maxj,maxk), rhoavg(0:maxi+1,0:maxj+1,0:maxk)
  REAL :: fx0avg(5,maxi,maxj), fwavg(4,maxi,maxj), windavg(4,maxi,maxj)
  REAL :: psisl(0:maxi,0:maxj,maxisles), ubisl(2,0:maxi+1,0:maxj,maxisles)
  REAL :: erisl(maxisles,maxisles+1), psibc(maxisles)
  REAL :: ts_store(maxl,maxi,maxj,maxk)
  REAL :: albcl(maxi,maxj)
  REAL, DIMENSION(maxi,maxj) :: &
       & evap_save1, late_save1, sens_save1, evap_save2, late_save2, sens_save2
  REAL, DIMENSION(maxi,maxj) :: &
       & mldpebuoy, mldpeconv, mldpelayer1, mldketau, mldemix, mld
  REAL, DIMENSION(maxk) :: mlddec, mlddecd
  INTEGER :: mldk(maxi,maxj)
  REAL :: ediff1(maxi,maxj,maxk-1), diffmax(maxk)
  REAL :: ssmax(maxk-1)
  LOGICAL :: getj(maxi,maxj)
\end{verbatim}

\section{Allocation and deallocation}


%======================================================================
\chapter{\texttt{sedgem}}

\section{Source files}

\begin{tabular}{rl}
 114 & \texttt{sedgem\_box\_benthic.f90} \\
 127 & \texttt{end\_sedgem.f90} \\
 146 & \texttt{cpl\_comp\_sedgem.f90} \\
 154 & \texttt{cpl\_flux\_sedgem.f90} \\
 208 & \texttt{initialise\_sedgem.f90} \\
 474 & \texttt{sedgem\_box\_ridgwelletal2003\_sedflx.f90} \\
 570 & \texttt{sedgem.f90} \\
 592 & \texttt{sedgem\_lib.f90} \\
1042 & \texttt{sedgem\_box\_archer1991\_sedflx.f90} \\
1573 & \texttt{sedgem\_data\_netCDF.f90} \\
2067 & \texttt{sedgem\_box.f90} \\
2412 & \texttt{sedgem\_data.f90} \\
\end{tabular}

Need to merge \texttt{sedgem.f90} \\ \texttt{initialise\_sedgem.f90}
and \texttt{end\_sedgem.f90}.

\section{Coordinate variables}

\begin{verbatim}
  INTEGER, PARAMETER :: n_i = ilon1_sed
  INTEGER, PARAMETER :: n_j = ilat1_sed
  INTEGER, PARAMETER :: n_phys_sed = 14
  INTEGER, PARAMETER :: n_opt_sed = 26
\end{verbatim}

\section{Array definitions}

Almost everything is already allocatable.  Should be pretty easy to
do.

%======================================================================
\chapter{\texttt{biogem}}

\section{Source files}

\begin{tabular}{rl}
  66 & \texttt{end\_biogem.f90} \\
  92 & \texttt{cpl\_comp\_biogem.f90} \\
 353 & \texttt{initialise\_biogem.f90} \\
1647 & \texttt{biogem\_lib.f90} \\
2470 & \texttt{biogem\_data\_ascii.f90} \\
2695 & \texttt{biogem.f90} \\
2994 & \texttt{biogem\_data\_netCDF.f90} \\
3146 & \texttt{biogem\_data.f90} \\
3867 & \texttt{biogem\_box.f90} \\
\end{tabular}

\section{Coordinate variables}

\begin{verbatim}
  INTEGER,PARAMETER::n_i = ilon1_ocn
  INTEGER,PARAMETER::n_j = ilat1_ocn
  INTEGER,PARAMETER::n_k = inl1_ocn
  INTEGER,PARAMETER::n_phys_ocn = 21
  INTEGER,PARAMETER::n_phys_ocnatm = 25
  INTEGER,PARAMETER::n_data_max = 32767
  INTEGER,PARAMETER::n_opt_misc = 14
  INTEGER,PARAMETER::n_opt_atm = 01
  INTEGER,PARAMETER::n_opt_bio = 06
  INTEGER,PARAMETER::n_opt_force = 08
  INTEGER,PARAMETER::n_opt_data = 30
  INTEGER,PARAMETER::n_opt_select = 05
  INTEGER,PARAMETER::n_diag_bio = 09
  INTEGER,PARAMETER::n_diag_geochem = 07
  INTEGER,PARAMETER::n_diag_misc_2D = 07
\end{verbatim}

\section{Array definitions}

Lots...

\section{Allocation and deallocation}

\end{document}
